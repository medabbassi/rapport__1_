% Options for packages loaded elsewhere
\PassOptionsToPackage{unicode}{hyperref}
\PassOptionsToPackage{hyphens}{url}
\documentclass[
]{article}
% Page layout for A4 paper
\usepackage[a4paper,margin=2.5cm,top=3cm,bottom=3cm]{geometry}
\usepackage{xcolor}
\usepackage{amsmath,amssymb}
\setcounter{secnumdepth}{-\maxdimen} % remove section numbering

% Enhanced packages for figures and tables
\usepackage{float}  % Improved float placement
\usepackage{caption}  % Enhanced captions
\usepackage{subcaption}  % Subcaptions for figures
\usepackage{adjustbox}  % Better image scaling
\usepackage{array}  % Enhanced table formatting
\usepackage{colortbl}  % Table coloring
\usepackage{hhline}  % Enhanced table lines

% Enable figure and table numbering
\setcounter{figure}{0}
\setcounter{table}{0}
\renewcommand{\thefigure}{\arabic{figure}}
\renewcommand{\thetable}{\arabic{table}}
\usepackage{iftex}
\ifPDFTeX
  \usepackage[T1]{fontenc}
  \usepackage[utf8]{inputenc}
  \usepackage{textcomp} % provide euro and other symbols
\else % if luatex or xetex
  \usepackage{unicode-math} % this also loads fontspec
  \defaultfontfeatures{Scale=MatchLowercase}
  \defaultfontfeatures[\rmfamily]{Ligatures=TeX,Scale=1}
\fi
\usepackage{lmodern}
\ifPDFTeX\else
  % xetex/luatex font selection
\fi
% Use upquote if available, for straight quotes in verbatim environments
\IfFileExists{upquote.sty}{\usepackage{upquote}}{}
\IfFileExists{microtype.sty}{% use microtype if available
  \usepackage[]{microtype}
  \UseMicrotypeSet[protrusion]{basicmath} % disable protrusion for tt fonts
}{}
\makeatletter
\@ifundefined{KOMAClassName}{% if non-KOMA class
  \IfFileExists{parskip.sty}{%
    \usepackage{parskip}
  }{% else
    \setlength{\parindent}{0pt}
    \setlength{\parskip}{6pt plus 2pt minus 1pt}}
}{% if KOMA class
  \KOMAoptions{parskip=half}}
\makeatother
\usepackage{longtable,booktabs, array}
\usepackage{multirow}
\usepackage{calc} % for calculating minipage widths

% Enhanced table formatting with borders
\renewcommand{\arraystretch}{1.2}  % Increase row spacing
\setlength{\arrayrulewidth}{0.5pt}  % Set border width
\setlength{\tabcolsep}{8pt}  % Increase column padding

% Correct order of tables after \paragraph or \subparagraph
\usepackage{etoolbox}
\makeatletter
\patchcmd\longtable{\par}{\if@noskipsec\mbox{}\fi\par}{}{}
\makeatother
% Allow footnotes in longtable head/foot
\IfFileExists{footnotehyper.sty}{\usepackage{footnotehyper}}{\usepackage{footnote}}
\makesavenoteenv{longtable}
\usepackage{graphicx}
\makeatletter
\newsavebox\pandoc@box
\newcommand*\pandocbounded[1]{% scales image to fit in text height/width
  \sbox\pandoc@box{#1}%
  \Gscale@div\@tempa{\textheight}{\dimexpr\ht\pandoc@box+\dp\pandoc@box\relax}%
  \Gscale@div\@tempb{\linewidth}{\wd\pandoc@box}%
  \ifdim\@tempb\p@<\@tempa\p@\let\@tempa\@tempb\fi% select the smaller of both
  \ifdim\@tempa\p@<\p@\scalebox{\@tempa}{\usebox\pandoc@box}%
  \else\usebox{\pandoc@box}%
  \fi%
}
% Set default figure placement to htbp
\def\fps@figure{htbp}
\makeatother
\setlength{\emergencystretch}{3em} % prevent overfull lines
\providecommand{\tightlist}{%
  \setlength{\itemsep}{0pt}\setlength{\parskip}{0pt}}
\usepackage{bookmark}
\IfFileExists{xurl.sty}{\usepackage{xurl}}{} % add URL line breaks if available
\urlstyle{same}
\hypersetup{
  hidelinks,
  pdfcreator={LaTeX via pandoc}}

\author{}
\date{}

\begin{document}

\begin{enumerate}
\def\labelenumi{\Roman{enumi}.}
\item
  \textbf{Introduction}
\end{enumerate}

\begin{enumerate}
\def\labelenumi{\arabic{enumi}.}
\item
  \textbf{Background and Context}
\end{enumerate}

In the field of software development, error handling represents a major
challenge. Bugs and anomalies, whether detected during the development
phase or in production, require rapid and effective intervention to
avoid service interruptions and financial losses. However, traditional
approaches rely on manual detection, complex diagnosis and often
time-consuming remediation, slowing down the development cycle.
Additionally, the integration of DevOps and CI/CD solutions remains a
manual process, requiring constant monitoring and human intervention.

\begin{enumerate}
\def\labelenumi{\arabic{enumi}.}
\setcounter{enumi}{1}
\item
  \textbf{Problem Statement}
\end{enumerate}

Errors and bugs are common in software development and can slow down
production cycles. Manually identifying, analysing, and correcting these
errors takes time, consumes resources, and can impact the quality of the
final product. Additionally, DevOps integration necessitates continuous
monitoring and manual interventions, which can make it challenging to
deploy applications quickly and securely.

ErrorZen is an intelligent platform that automatically detects, analyses
and fixes errors in backend, frontend and mobile applications. Thanks to
artificial intelligence and advanced DevOps integration, ErrorZen:\\
✅ Identifies errors and anomalies in real time.\\
✅ Offers AI-based auto-correction solutions.\\
✅ Integrates CI/CD pipelines to automate testing and deployments.\\
✅ Notify developers in real-time via Slack, email, or webhook.

This solution helps reduce bug-fixing time, improve application quality,
and speed up the development cycle.

\begin{enumerate}
\def\labelenumi{\arabic{enumi}.}
\setcounter{enumi}{2}
\item
  \textbf{Objectives of the Project}
\end{enumerate}

This project aims to design and develop ErrorZen, an intelligent
platform for automating error management in web and mobile applications.
The main objectives are:

\begin{itemize}
\item
  Automatic error detection in real time on different platforms
  (frontend, backend, mobile).
\item
  Automated analysis and correction of anomalies using artificial
  intelligence models.
\item
  Advanced DevOps integration, enabling automation of testing and
  deployment after patching.
\item
  Centralisation and visualisation of errors via an interactive
  dashboard, optimised with   and gRPC for effective
  communication.
\item
  Instant notification to development teams via tools such as Slack,
  email or webhooks.
\end{itemize}

\begin{enumerate}
\def\labelenumi{\arabic{enumi}.}
\setcounter{enumi}{3}
\item
  \textbf{Agile Methodology: Why Scrum/RAD?}
\end{enumerate}

The project will follow a RAD (Rapid Application Development) approach
to ensure rapid and iterative development and will respect the
Agile-Scrum development cycle. The platform will be built with:

\begin{itemize}
\item
  Backend: Go, PostgreSQL, gRPC/Rest API for
  communication between services.
\item
  Frontend: Vue.js with   Client for integration.
\item
  Artificial Intelligence: Models based on deepseek api for
  automatic error correction.
\item
  DevOps: CI/CD via GitHub Actions/Jenkins, deployment with Docker, and
  Kubernetes on AWS
\end{itemize}

Expected Results

\begin{itemize}
\item
  A significant reduction in error correction time in the development
  cycle.
\item
  Improved application reliability through self-correction and automated
  testing.
\item
  Acceleration of the production process via DevOps automation.
\item
  An intuitive interface allows developers to track and manage errors in
  real time.
\end{itemize}

ErrorZen will provide an innovative solution to optimise error handling
and automate DevOps cycles, thereby reducing developer workload and
improving software quality.

\begin{enumerate}
\def\labelenumi{\arabic{enumi}.}
\setcounter{enumi}{4}
\item
  \textbf{Overview of Sprints}
\end{enumerate}

The project was divided into \textbf{7 sprints}, each lasting
\textbf{two weeks}:

\begin{itemize}
\item
  \textbf{Sprint 1}: Project setup, requirements gathering, and initial
  architecture.
\item
  \textbf{Sprint 2}: Development of core modules (e.g., user
  authentication, data models).
\item
  \textbf{Sprint 3}: Integration with back-end, user interface refinements, testing.
\item
  \textbf{Sprint 4}: Final features, deployment, documentation, and user
  training.
\end{itemize}

Each sprint had its backlog, planning session, and review meeting. This
structure helped ensure that the development process was both flexible
and measurable, with tangible results delivered at each iteration.

\textbf{II. Project Management \& Methodology}

\begin{enumerate}
\def\labelenumi{\arabic{enumi}.}
\item
  \textbf{Overview of Agile and Scrum}
\end{enumerate}

Agile is a flexible software development methodology that emphasises
iterative progress, collaboration, and user feedback. Scrum is a widely
used Agile framework characterised by short, time-boxed development
cycles called sprints, daily team meetings, and continuous delivery of
value.

In this project, Scrum was adopted to manage changing requirements and
ensure a structured yet adaptable development process.

\begin{enumerate}
\def\labelenumi{\arabic{enumi}.}
\setcounter{enumi}{1}
\item
  \textbf{Adapting Scrum Roles in a RAD Context:}
\end{enumerate}

\begin{quote}
In a traditional Scrum team, roles are clearly defined:
\end{quote}

\begin{itemize}
\item
  Product Owner: Represents the client, prioritises the product backlog,
  and validates features.
\item
  Scrum Master: Ensures adherence to Agile principles, removes blockers,
  and facilitates ceremonies.
\item
  Development Team: Delivers functional increments each sprint.
\end{itemize}

My RAD (Rapid Application Development) Adaptation (Solo/Small Team):

Working in a fast-paced, iterative RAD environment (where speed and
client feedback are critical), I merged these roles to maximise
efficiency:

\begin{itemize}
\item
  Product Owner + RAD Analyst
\item
  Direct client collaboration to capture just-in-time requirements and
  dynamically adjust the backlog.
\item
  MVP-driven prioritisation (typical of RAD cycles), with client
  feedback integrated after each prototype.
\item
  Use of visual tools (e.g., Miro, clickable mockups) for rapid
  requirement validation.
\item
  Scrum Master + Technical Coordinator
\item
  Self-facilitated ceremonies (e.g., solo planning poker via relative
  effort points, retrospectives focused on continuous improvement).
\item
  Proactive risk/blocker management with strict timeboxing (e.g.,
  capping technical spikes at 2 hours).
\item
  Leveraged RAD tools (low-code platforms, code generators) to
  accelerate development cycles.
\item
  Full-Stack Developer + Integrator
\item
  Feature-driven implementation with technical spikes for Proof of
  Concepts (POCs).
\item
  Automated testing and CI/CD pipelines for real-time validation of each
  increment.
\item
  Incremental documentation via living docs (e.g., updated README.md per
  commit).
\item
  Benefits of This Hybrid Approach:
\item
  Faster time-to-market by eliminating role synchronisation delays.
\item
  Greater flexibility to pivot based on client feedback (core RAD
  principle).
\item
  Stronger ownership across the entire development lifecycle.
\end{itemize}

\begin{enumerate}
\def\labelenumi{\arabic{enumi}.}
\setcounter{enumi}{2}
\item
  \textbf{Scrum Ceremonies}
\end{enumerate}

\begin{quote}
The following ceremonies were adopted:
\end{quote}

\begin{itemize}
\item
  \textbf{Sprint Planning}: Defined sprint goals and selected backlog
  items.
\item
  \textbf{Daily Stand-ups}: Brief updates on progress and blockers
  (documented if solo).
\item
  \textbf{Sprint Review}: Demonstrated completed features to
  stakeholders or self-evaluated.
\item
  \textbf{Sprint Retrospective}: Identified improvements to apply in the
  next sprint.
\end{itemize}

\begin{enumerate}
\def\labelenumi{\arabic{enumi}.}
\setcounter{enumi}{3}
\item
  \textbf{Tools Used}
\end{enumerate}

\begin{itemize}
\item
  Project Management: Trello (backlog, sprint boards)
\item
  Version Control: Git + GitHub
\item
  Documentation: Notion / Google Docs
\item
  Design \& Wireframing: Figma / Draw.io
\item
  Code Editor: VS Code / Android Studio / GoLand /PgAdmin/Docker Desktop
\end{itemize}

\begin{enumerate}
\def\labelenumi{\arabic{enumi}.}
\setcounter{enumi}{4}
\item
  \textbf{Sprint Length and Structure }
\end{enumerate}

Each sprint lasted \textbf{two weeks} and followed this structure:

\begin{itemize}
\item
  Day 1: Sprint Planning
\item
  Day 2--12: Development + Daily Stand-ups
\item
  Day 13: Sprint Review (demo or milestone check)
\item
  Day 14: Sprint Retrospective
\end{itemize}

\begin{enumerate}
\def\labelenumi{\arabic{enumi}.}
\setcounter{enumi}{5}
\item
  \textbf{Project Timeline}
\end{enumerate}

A total of \textbf{7 sprints} were conducted, as shown in the timeline
below:

\begin{table}[H]
\centering
\caption{Sprint Timeline and Goals}
\label{tab:sprints}
\small
\begin{tabular}{|p{2cm}|p{3cm}|p{8cm}|}
\hline
\textbf{Sprint} & \textbf{Duration} & \textbf{Goal} \\ \hline
Sprint 1 & Week 1--2 & Core Infrastructure Setup: Backend (Go/gRPC), PostgreSQL, CI/CD pipeline. \\ \hline
Sprint 2 & Week 3--4 & Error Ingestion \& Dashboard MVP: Real-time error capture + Vue.js UI. \\ \hline
Sprint 3 & Week 5--6 & AI Integration: PyTorch model training for error classification. \\ \hline
Sprint 4 & Week 7--8 & Auto-Correction Engine: AI-driven fixes + unit test generation. \\ \hline
Sprint 5 & Week 9--10 & DevOps Automation: GitHub Actions workflows, K8S deployment. \\ \hline
Sprint 6 & Week 11--12 & SDK \& Integrations: Sentry/Firebase compatibility, Slack alerts. \\ \hline
Sprint 7 & Week 13--14 & Polish \& Scalability: Load testing, security audits, documentation. \\ \hline
\end{tabular}
\end{table}

\begin{enumerate}
\def\labelenumi{\arabic{enumi}.}
\setcounter{enumi}{6}
\item
  \textbf{Gantt}
\end{enumerate}

\begin{figure}[H]
\centering
\includegraphics[width=0.9\textwidth,height=0.6\textheight,keepaspectratio]{media/image4.png}
\caption{Gantt Chart - Project Timeline}
\label{fig:gantt}
\end{figure}

\textbf{III.} \textbf{Literature Review / State of the Art}

\begin{enumerate}
\def\labelenumi{\arabic{enumi}.}
\item
  \textbf{Introduction}
\end{enumerate}

Before implementing any technical solution, it is essential to review
existing work, approaches, and technologies related to the problem being
addressed. This review of the literature aims to provide an overview of similar
systems, tools, and frameworks and to justify the technical choices
made during this project.

\begin{enumerate}
\def\labelenumi{\arabic{enumi}.}
\setcounter{enumi}{1}
\item
  Existing Solutions
\end{enumerate}

Several platforms and tools have been developed to address error/bug
logging and detection. Each offers different features and uses various
technologies.\\
\strut \\
A- Sentry\\
\begin{figure}[H]
\centering
\includegraphics[width=0.8\textwidth,keepaspectratio]{media/image1.png}
\caption{System Architecture Overview}
\label{fig:architecture}
\end{figure}

\begin{figure}[H]
\centering
\includegraphics[width=0.7\textwidth,keepaspectratio]{media/image3.png}
\caption{System Design Diagram}
\label{fig:design}
\end{figure}

The Sentry platform is a popular error monitoring and performance
tracking tool used by developers to diagnose, fix, and optimise
applications. Below are some of its \textbf{pros} and \textbf{cons}:

Pros

\begin{itemize}
\item
  Real-Time Error Monitoring -- Quickly detects and alerts developers
  about crashes and exceptions.
\item
  Wide Language Support -- Works with JavaScript,  , Ruby, Java,
  Go, PHP, .NET, and more.
\item
  Detailed Error Reports -- Provides stack traces, environment data, and
  user context for debugging.
\item
  Performance Monitoring -- Tracks latency, slow transactions, and
  bottlenecks in applications.
\item
  Integrations -- Supports GitHub, Slack, Jira, and other DevOps tools
  for streamlined workflows.
\item
  Open-Source Option -- A self-hosted version is available for greater
  control over data.
\item
  User-Friendly UI -- Intuitive dashboard with filtering and search
  capabilities.
\item
  Release Tracking -- Correlates errors with specific code deployments.
\end{itemize}

Cons

\begin{itemize}
\item
  Cost for High Volume -- Can become expensive for large-scale
  applications with many events.
\item
  Limited Free Tier -- The free plan has restricted features and event
  limits.
\item
  Complex Setup for Self-Hosting -- Requires maintenance and
  infrastructure if deployed on-premise.
\item
  No Built-In APM (Advanced Performance Monitoring) -- Lags behind
  competitors like Datadog in full APM capabilities.
\item
  Steep Learning Curve -- Some features (like performance tracing) may
  require deeper configuration.
\item
  Limited Log Management -- Primarily focused on errors, not a full log
  analytics solution.
\end{itemize}

B- Dynatrace

\begin{figure}[H]
\centering
\includegraphics[width=0.8\textwidth,keepaspectratio]{media/image5.png}
\caption{Database Schema Design}
\label{fig:database}
\end{figure}

Dynatrace is an AI-powered, full-stack observability platform that
provides application performance monitoring (APM), infrastructure
monitoring, real-user monitoring (RUM), and cloud automation. It's known
for its automatic and intelligent insights, making it a favourite for
enterprises. Below are some of its \textbf{pros} and \textbf{cons}:

\begin{itemize}
\item
  AI-Powered Root Cause Analysis (Davis AI)
\item
  Automatically detects anomalies, pinpoints failures, and suggests
  fixes.
\item
  Significantly reduces mean time to resolution (MTTR).
\item
  Full-Stack Observability
\item
  Tracks applications, microservices, containers, cloud infra,
  databases, and network performance in one place.
\item
  Supports OpenTelemetry, Prometheus, and Kubernetes natively.
\item
  Automatic Discovery \& Dependency Mapping
\item
  Dynamically maps application dependencies without manual
  configuration.
\item
  Real User Monitoring (RUM) \& Synthetic Monitoring
\item
  Tracks real user experience (browser/mobile) and simulates synthetic
  transactions.
\item
  Cloud-Native \& Multi-Cloud Support
\item
  Works seamlessly with AWS, Azure, GCP, and hybrid environments.
\end{itemize}

Cons:

\begin{itemize}
\item
  Very Expensive
\item
  One of the most costly APM tools, making it less accessible for SMBs.
\item
  Complex Setup \& Learning Curve
\item
  Overwhelming for beginners due to its advanced features.
\item
  Requires training to utilise AI-driven insights fully.
\item
  Limited Customisation in Dashboards
\item
  Some users find dashboarding less flexible compared to Grafana or
  Datadog.
\item
  Heavy Resource Consumption (On-Premise)
\item
  Self-hosted deployments require significant infrastructure.
\item
  No Built-In Log Management (Requires Grail)
\item
  Log analytics is a separate module (Dynatrace Grail), adding to costs.
\item
  Vendor Lock-In Risk
\item
  Proprietary agents and data models make migration difficult.
\end{itemize}

\begin{table}[H]
\centering
\caption{Comparison of Sentry vs Dynatrace}
\footnotesize
\begin{tabular}{|p{4.5cm}|p{4.5cm}|p{4.5cm}|}
\hline
\textbf{Feature/Capability} & \textbf{Sentry} & \textbf{Dynatrace} \\ \hline
Primary Use Case & Error \& Performance Monitoring & Full-Stack APM \& AI Observability \\ \hline
Scalability & Poor at large-scale event volumes & Highly scalable (enterprise-grade) \\ \hline
Offline Support & No offline error tracking & No offline monitoring \\ \hline
User Interface (UI) & Modern but simple & Powerful but complex \\ \hline
Key Missing Features & No infra/cloud monitoring & No built-in log management (Grail add-on) \\ \hline
Root Cause Analysis & Manual (basic traces) & AI-powered (Davis AI) \\ \hline
Real User Monitoring & Limited (frontend-focused) & Advanced (RUM + Synthetic) \\ \hline
Performance Monitoring & Basic (transactions, latency) & Full APM (code-level, DB, infra) \\ \hline
Cloud/Serverless & Limited & AWS Lambda, Azure Functions, etc. \\ \hline
Cost & Affordable for startups & Very expensive (enterprise pricing) \\ \hline
Best For & Dev teams need error tracking & Enterprises needing AI-driven APM \\ \hline
\end{tabular}
\end{table}

Conclusion: Why ErrorZen will Outperform Existing Solutions

This analysis of Sentry and Dynatrace highlights critical gaps in
current error monitoring and observability tools, reinforcing the need
for a custom, AI-driven solution like ErrorZen. While traditional
platforms excel in specific areas (Sentry for error tracking, Dynatrace
for APM), they suffer from:

\begin{itemize}
\item
  Limited automation (manual triaging, no auto-fixing).
\item
  Poor scalability for high-frequency errors.
\item
  No seamless DevOps/CI/CD integration (requiring manual intervention).
\item
  Lack of AI-powered remediation is delaying root cause analysis.
\item
  How ErrorZen Solves These Challenges
\item
  AI-Powered Auto-Correction
\item
  Unlike Sentry (manual debugging) or Dynatrace (AI alerts only),
  ErrorZen proactively fixes errors using machine learning, slashing
  MTTR.
\item
  End-to-End DevOps Automation
\item
  Integrates directly with CI/CD pipelines to auto-test and deploy
  patches, eliminating manual steps that Dynatrace and Sentry can't
  address.
\item
  Unified Cross-Platform Monitoring
\item
  Tracks frontend, backend, and mobile in one dashboard, while
  competitors silo data (e.g., Sentry lacks infra insights).
\item
  Real-Time Notifications \& Collaboration
\item
  Combines Slack/email alerts with actionable fixes, unlike passive
  Dynatrace alerts or Sentry's basic notifications.
\item
  Scalable \& Cost-Effective
\item
  Avoid Dynatrace's enterprise pricing and Sentry's volume limits via
  optimised event processing.
\item
  The Future of Error Management
\item
  ErrorZen doesn't just monitor---it resolves. By merging AI-driven
  diagnostics, automated remediation, and deep DevOps integration, it
  bridges the gaps left by legacy tools, offering:
\item
  Faster deployments (reduced manual reviews).
\item
  Higher reliability (auto-patched errors).
\item
  Lower costs (no overprovisioning for scale).
\item
  For teams prioritising speed, automation, and intelligence, ErrorZen
  isn't an alternative---it's the next evolution.
\end{itemize}

\textbf{2. Technology Choices Justification}

The platform will be built with:

\begin{itemize}
\item
  Backend: Go/ / , PostgreSQL/ , gRPC/Rest API for
  communication between services.
\item
  Frontend: Vue.js with   Client for   integration.
\item
  Artificial Intelligence: Models based on deepseek api for
  automatic error correction.
\item
  DevOps: CI/CD via GitHub Actions/Jenkins, deployment with Docker and
  Kubernetes on AWS
\end{itemize}

\begin{table}[H]
\centering
\caption{Technology Stack Justification}
\footnotesize
\begin{tabular}{|p{3cm}|p{3.5cm}|p{7.5cm}|}
\hline
\textbf{Component} & \textbf{Technology Selected} & \textbf{Why Chosen? (Advantages Over Alternatives)} \\ \hline
Backend Language & Go (Primary) & Go: High performance, concurrency (ideal for real-time error processing). Python: Async I/O for event-driven tasks, AI/ML integration ease. \\ \hline
Database & PostgreSQL (Primary), MongoDB & PostgreSQL: ACID compliance, scalability, JSON support. MongoDB: Flexible schema for unstructured error logs. \\ \hline
API Communication & gRPC (Internal), REST (External) & gRPC: Low-latency, high-throughput microservices communication. REST: Simplicity for client integrations. \\ \hline
Frontend & Vue.js + GraphQL Client & Vue.js: Lightweight, reactive UI for dashboards. GraphQL Client: Efficient state management. \\ \hline
AI/ML Framework & PyTorch (Primary), TensorFlow & PyTorch: Dynamic graphs, better for iterative AI model tuning. TensorFlow: Backup for production-scale deployments. \\ \hline
DevOps CI/CD & GitHub Actions (Primary), Jenkins (Legacy) & GitHub Actions: Native Git integration, faster workflows. Jenkins: Fallback for complex pipelines. \\ \hline
Deployment & Docker + Kubernetes (AWS/GCP) & Docker: Consistency across environments. Kubernetes: Auto-scaling for error spike handling. AWS/GCP: Global reliability. \\ \hline
\end{tabular}
\end{table}

\subsubsection{\texorpdfstring{\textbf{3.
Summary}}{3. Summary}}\label{summary}

The literature and technology review provided essential insights into:

\begin{itemize}
\item
  The current state of the market
\item
  Common limitations in existing systems
\item
  Best practices in selecting modern, scalable technologies
\end{itemize}

This helped shape a solution that is both technically sound and aligned
with the client\textquotesingle s real-world needs.

\textbf{IV. Requirement Analysis}

\begin{enumerate}
\def\labelenumi{\arabic{enumi}.}
\item
  \textbf{Introduction}
\end{enumerate}

This chapter outlines the system\textquotesingle s requirements,
including both functional and non-functional aspects. It is the
foundation upon which the system\textquotesingle s design and
implementation are based. The requirements were collected through
meetings with stakeholders, analysis of the domain, and study of
existing systems.

\begin{enumerate}
\def\labelenumi{\arabic{enumi}.}
\setcounter{enumi}{1}
\item
  \textbf{Client Expectations}
\end{enumerate}

These requirements describe the main functionalities that the system
must offer.

\textbf{Error detection and handling}

\begin{itemize}
\item
  Capture real-time errors from the frontend, backend and mobile.
\item
  Centralise errors in an interactive dashboard.
\item
  Filter and categorise errors according to their criticality.
\end{itemize}

\textbf{AI error analysis and correction}

\begin{itemize}
\item
  Automatically analyse detected errors.
\item
  Propose adapted solutions based on artificial intelligence.
\item
  Generate unit tests to validate corrections before their integration.
\end{itemize}

\textbf{DevOps Automation}

\begin{itemize}
\item
  Trigger automated tests after correction.
\item
  Deploy patched code through a CI/CD pipeline without manual
  intervention.
\item
  Ensure follow-up of corrections and production releases.
\end{itemize}

\textbf{Integration with other tools}

\begin{itemize}
\item
  Provide an SDK and plugins to integrate ErrorZen with other
  technologies (e.g. Firebase Crashlytics, Sentry).
\item
  Offer an API to allow businesses to customise the integration.
\end{itemize}

\textbf{User interface and access management}

\begin{itemize}
\item
  Allow developers to view, filter and analyse errors via a dashboard.
\item
  Manage roles and permissions to secure access to data.
\end{itemize}

\begin{enumerate}
\def\labelenumi{\arabic{enumi}.}
\setcounter{enumi}{2}
\item
  \textbf{Non-functional Requirements}
\end{enumerate}

These needs concern system quality, performance and security.

\textbf{Performance and scalability}

\begin{itemize}
\item
  Manage a large volume of logs without slowing down.
\item
  Ensure rapid system response (\textless200ms for error recovery).
\item
  Support a large number of users and integrations without loss of
  performance.
\end{itemize}

\textbf{Security and Compliance}

\begin{itemize}
\item
  Encrypt sensitive user and error data.
\item
  Authenticate and authorise access via OAuth or JWT.
\item
  Ensure compliance with security standards (e.g. GDPR, ISO 27001).
\end{itemize}

\textbf{Availability and reliability}

\begin{itemize}
\item
  Ensure high availability (SLA \textgreater{} 99.9\%).
\item
  Implement a backup and recovery mechanism in the event of a failure.
\item
  Have real-time monitoring to prevent any failure.
\end{itemize}

\textbf{Compatibility and integration}

\begin{itemize}
\item
  Support different environments (Linux, Windows, macOS).
\item
  Ensure compatibility with several languages
  \hspace{0pt}\hspace{0pt}and frameworks ( ,  , Java,
  Flutter, etc.).
\item
  Use   for efficient communication with the frontend.
\end{itemize}

\textbf{Ease of use and maintainability}

\begin{itemize}
\item
  Offer an intuitive and accessible interface.
\item
  Document the API and SDKs for easy integration.
\item
  Provide technical support and regular updates.
\end{itemize}

\begin{enumerate}
\def\labelenumi{\arabic{enumi}.}
\setcounter{enumi}{3}
\item
  \textbf{Constraints}
\end{enumerate}

\begin{itemize}
\item
  Time-to-Market
\item
  The MVP must be delivered in 17 weeks to meet client onboarding
  deadlines.
\item
  Rationale: Rapid Application Development (RAD) and Agile-Scrum
  methodologies will accelerate iterations.
\item
  Offline-First Support
\item
  Must cache and sync errors locally for mobile/remote developers with
  poor connectivity.
\item
  Rationale: PostgreSQL's write-ahead logging (WAL) and Vue.js's local
  storage ensure data consistency.
\item
  Open-Source Priority
\item
  Prefer open-source tools (e.g., PostgreSQL, PyTorch) to minimise
  licensing costs.
\item
  Rationale: Aligns with non-functional need for cost-effective
  scalability.
\item
  Multi-Platform SDKs
\item
  SDKs must support  ,  , Java, and mobile (iOS/Android) for
  broad compatibility.
\item
  Rationale: Ensures seamless integration with diverse client stacks
  (functional need: "Integration with other tools").
\item
  Zero Manual DevOps
\item
  CI/CD pipelines (GitHub Actions/Jenkins) must fully automate
  testing/deployment without human intervention.
\item
  Rationale: Addresses the problematic emphasis on eliminating manual
  DevOps processes.
\end{itemize}

\begin{enumerate}
\def\labelenumi{\arabic{enumi}.}
\setcounter{enumi}{4}
\item
  \textbf{Diagrams}
\end{enumerate}

\begin{enumerate}
\def\labelenumi{\alph{enumi}.}
\item
  \textbf{Uses-case diagram}
\end{enumerate}

\begin{figure}[H]
\centering
\includegraphics[width=0.7\textwidth,keepaspectratio]{rapport/media/userscasglobal.png}
\caption{Use Case Diagram}
\label{fig:usecase}
\end{figure}

\begin{enumerate}
\def\labelenumi{\alph{enumi}.}
\setcounter{enumi}{1}
\item
  \textbf{Class diagram of the system}
\end{enumerate}

\begin{figure}[H]
\centering
\includegraphics[width=\textwidth,keepaspectratio]{rapport/media/diag_class_global.png}
\caption{Class Diagram of the System}
\label{fig:classdiagram}
\end{figure}

\begin{enumerate}
\def\labelenumi{\alph{enumi}.}
\setcounter{enumi}{2}
\item
  Deployment diagram
\end{enumerate}

\begin{figure}[H]
\centering
\includegraphics[width=0.6\textwidth,keepaspectratio]{rapport/media/diag_deploy.png}
\caption{Deployment Architecture}
\label{fig:components}
\end{figure}

\begin{enumerate}
\def\labelenumi{\alph{enumi}.}
\setcounter{enumi}{3}
\item
  Requirement Traceability Matrix

  \begin{table}[H]
  \centering
  \caption{Requirement Traceability Matrix}
  \footnotesize
  \begin{tabular}{|p{1.5cm}|p{5cm}|p{2.5cm}|p{3cm}|p{2cm}|}
    \hline
    \textbf{Req. ID} & \textbf{Requirement Description} & \textbf{Source} & \textbf{Implementation Module} & \textbf{Status} \\
    \hline
    RQ-01 & The system must authenticate all users before access & Business Rule & Auth Module & Implemented \\
    \hline
    RQ-02 & Developer must handle errors & Functional Req. & Error Handling Service & In Testing \\
    \hline
    RQ-03 & Developer must handle analytics & Functional Req. & Analytics Service & Implemented \\
    \hline
    RQ-04 & AI agent must capture real-time errors & Functional Req. & AI Monitoring Module & Pending \\
    \hline
    RQ-05 & AI agent must suggest possible fixes & Functional Req. & AI Recommendation Engine & Planned \\
    \hline
    RQ-06 & System must generate bills automatically & Functional Req. & Billing Service & Implemented \\
    \hline
    RQ-07 & Admin must manage user roles & Functional Req. & User Management Module & Implemented \\
    \hline
    RQ-08 & Manager must monitor deployment processes & Functional Req. & Deployment Service & In Testing \\
    \hline
  \end{tabular}
  \end{table}

\end{enumerate}

\textbf{6. Summary}

This chapter defined the expected functionalities and performance
characteristics of the system. These requirements guided the design and
implementation of the application. The use of diagrams helped visualise
user interactions and data flows clearly.


\textbf{V. Design and Architecture\\
1. Introduction}

This chapter presents the overall architecture of the system, the main
design decisions taken, the technologies and tools used, and the UML
diagrams that describe the internal structure and behaviour of the
system. The objective is to ensure the system is modular, scalable,
maintainable, and aligned with the requirements defined in the previous
chapter\textbf{.\\
\strut \\
2. System architecture}

\begin{figure}[H]
\centering
\includegraphics[width=0.8\textwidth,keepaspectratio]{rapport/media/communication_diag.png}
\caption{Detailed System Flow}
\label{fig:systemflow}
\end{figure}

\textbf{3. Database Design}

\begin{figure}[H]
\centering
\includegraphics[width=\textwidth,keepaspectratio]{rapport/media/erd.png}
\caption{Complete Database Design and ER Diagram}
\label{fig:erdatabase}
\end{figure}

\textbf{4. Design Principles}

The project followed key software engineering principles:

\begin{itemize}
\item
  Modularity: Code is divided into reusable components and services.
\item
  Separation of Concerns: Frontend, backend, and data layers are
  separated.
\item
  Security by Design: All API calls are authenticated; sensitive data is
  encrypted.
\item
  Scalability: Designed to scale horizontally by separating backend and
  database services.
\end{itemize}

5. Product Backlog

The product backlog was organized into 7 main epics, each containing multiple user stories distributed across the project sprints:

\begin{table}[H]
\centering
\caption{Product Backlog Summary by Epic}
\footnotesize
\begin{tabular}{|p{4cm}|p{2cm}|p{7cm}|}
\hline
\textbf{Epic} & \textbf{Sprint} & \textbf{Key User Stories} \\ \hline
Backend \& Data Auth & 1 & Setup Go/gRPC backend, PostgreSQL with WAL, Authentication UI, RBAC implementation \\ \hline
Real-Time Error Capture & 2 & Dashboard metrics UI, Error/Logs UI, API development, Backend integration \\ \hline
DevOps Foundation & 3 & Pipeline dashboard, API development, CI/CD tools, Pipeline automation \\ \hline
Error Classification \& AI Fixes & 4 & AI model integration, Error tagging, Automated fixes, Unit test generation \\ \hline
Alerting \& Notifications & 5 & Tool integrations, Notification system, Billing alerts, Alert throttling \\ \hline
Data Protection \& Payments & 6 & AES-256 encryption, GDPR compliance, Usage computation, Payment services \\ \hline
SDKs \& Plugins & 7 & Node.js plugin, Flutter/Dart SDK, Service activation, Documentation \\ \hline
\end{tabular}
\end{table}

The complete backlog contained 35 user stories with estimated efforts ranging from 8 to 24 hours per story, totaling approximately 420 hours of development work across the 7 sprints.

\textbf{5. Summary}

This chapter outlines the system\textquotesingle s architecture and
design decisions. Technologies were selected to optimise development
speed, scalability, and maintainability. UML diagrams and ER models
supported a clear technical structure that guided the implementation
phase.

\textbf{VI. Sprints}

\begin{enumerate}
\def\labelenumi{\arabic{enumi}.}
\item
  \textbf{Sprint 1: Project Setup \& Initial Design}
\end{enumerate}

\begin{quote}
\textbf{1.1 Sprint Duration}

Start date: March 25, 2025

End date: April 7, 2025

Sprint length: 2 weeks

1.2 Sprint Goal

The primary objective of this sprint was to establish the project
structure, select the technology stack, create initial wireframes, and
define the initial version of the product backlog.
\end{quote}

{\def\LTcaptype{} % do not increment counter
\begin{longtable}[]{@{}
  >{\raggedright\arraybackslash}p{(\linewidth - 8\tabcolsep) * \real{0.1194}}
  >{\raggedright\arraybackslash}p{(\linewidth - 8\tabcolsep) * \real{0.3788}}
  >{\raggedright\arraybackslash}p{(\linewidth - 8\tabcolsep) * \real{0.0627}}
  >{\raggedright\arraybackslash}p{(\linewidth - 8\tabcolsep) * \real{0.3486}}
  >{\raggedright\arraybackslash}p{(\linewidth - 8\tabcolsep) * \real{0.0905}}@{}}
\toprule\noalign{}
\begin{minipage}[b]{\linewidth}\centering
UserStoryId
\end{minipage} & \begin{minipage}[b]{\linewidth}\centering
UserStory
\end{minipage} & \begin{minipage}[b]{\linewidth}\centering
TaskId
\end{minipage} & \begin{minipage}[b]{\linewidth}\raggedright
Task
\end{minipage} & \begin{minipage}[b]{\linewidth}\centering
Est. effort
\end{minipage} \\
\multirow{5}{=}{\begin{minipage}[b]{\linewidth}\centering
US001
\end{minipage}} &
\multirow{5}{=}{\begin{minipage}[b]{\linewidth}\centering
Set up Go/gRPC /RestfulApi backend

for time error ingestion
\end{minipage}} & \begin{minipage}[b]{\linewidth}\centering
T1.1
\end{minipage} & \begin{minipage}[b]{\linewidth}\raggedright
Initialise Go module and workspace
\end{minipage} & \begin{minipage}[b]{\linewidth}\centering
3
\end{minipage} \\
& & \begin{minipage}[b]{\linewidth}\centering
T1.2
\end{minipage} & \begin{minipage}[b]{\linewidth}\raggedright
Set up a gRPC server and define a proto

files
\end{minipage} & \begin{minipage}[b]{\linewidth}\centering
3
\end{minipage} \\
& & \begin{minipage}[b]{\linewidth}\centering
T1.3
\end{minipage} & \begin{minipage}[b]{\linewidth}\raggedright
Create a basic REST API router
\end{minipage} & \begin{minipage}[b]{\linewidth}\centering
2
\end{minipage} \\
& & \begin{minipage}[b]{\linewidth}\centering
T1.4
\end{minipage} & \begin{minipage}[b]{\linewidth}\raggedright
Add error logging middleware

(e.g., interceptors, logging libs)
\end{minipage} & \begin{minipage}[b]{\linewidth}\centering
5
\end{minipage} \\
& & \begin{minipage}[b]{\linewidth}\centering
T1.5
\end{minipage} & \begin{minipage}[b]{\linewidth}\raggedright
Test local gRPC and REST endpoints

using Postman and grpcurl
\end{minipage} & \begin{minipage}[b]{\linewidth}\centering
3
\end{minipage} \\
\multirow{5}{=}{\begin{minipage}[b]{\linewidth}\centering
US002
\end{minipage}} &
\multirow{5}{=}{\begin{minipage}[b]{\linewidth}\centering
Implement PostgreSQL for structured error

storagewith WAL(Write-Ahead logging)
\end{minipage}} & \begin{minipage}[b]{\linewidth}\centering
T2.1
\end{minipage} & \begin{minipage}[b]{\linewidth}\raggedright
Install and configure PostgreSQL locally
\end{minipage} & \begin{minipage}[b]{\linewidth}\centering
1
\end{minipage} \\
& & \begin{minipage}[b]{\linewidth}\centering
T2.2
\end{minipage} & \begin{minipage}[b]{\linewidth}\raggedright
Design an initial schema for error logs

(e.g., errors, services, projects)
\end{minipage} & \begin{minipage}[b]{\linewidth}\centering
5
\end{minipage} \\
& & \begin{minipage}[b]{\linewidth}\centering
T2.3
\end{minipage} & \begin{minipage}[b]{\linewidth}\raggedright
Configure Write-Ahead Logging (WAL)

for safe/error-tolerant write operations
\end{minipage} & \begin{minipage}[b]{\linewidth}\centering
2
\end{minipage} \\
& & \begin{minipage}[b]{\linewidth}\centering
T2.4
\end{minipage} & \begin{minipage}[b]{\linewidth}\raggedright
Create a migration script for the schema

initialisation using Go
\end{minipage} & \begin{minipage}[b]{\linewidth}\centering
4
\end{minipage} \\
& & \begin{minipage}[b]{\linewidth}\centering
T2.5
\end{minipage} & \begin{minipage}[b]{\linewidth}\raggedright
Write DB connection logic in Go with

retry and health-check capabilities
\end{minipage} & \begin{minipage}[b]{\linewidth}\centering
2
\end{minipage} \\
\multirow{3}{=}{\begin{minipage}[b]{\linewidth}\centering
US003
\end{minipage}} &
\multirow{3}{=}{\begin{minipage}[b]{\linewidth}\centering
Design RestApi for external integration
\end{minipage}} & \begin{minipage}[b]{\linewidth}\centering
T3.1
\end{minipage} & \begin{minipage}[b]{\linewidth}\raggedright
Define OpenAPI spec (Swagger) for

public-facing endpoints
\end{minipage} & \begin{minipage}[b]{\linewidth}\centering
4
\end{minipage} \\
& & \begin{minipage}[b]{\linewidth}\centering
T3.2
\end{minipage} & \begin{minipage}[b]{\linewidth}\raggedright
Implement one sample REST endpoint
\end{minipage} & \begin{minipage}[b]{\linewidth}\centering
1
\end{minipage} \\
& & \begin{minipage}[b]{\linewidth}\centering
T3.3
\end{minipage} & \begin{minipage}[b]{\linewidth}\raggedright
Add basic input validation and error

handling
\end{minipage} & \begin{minipage}[b]{\linewidth}\centering
2
\end{minipage} \\
\multirow{4}{=}{\begin{minipage}[b]{\linewidth}\centering
US004
\end{minipage}} &
\multirow{4}{=}{\begin{minipage}[b]{\linewidth}\centering
Implement Authentification UI with VueJs
\end{minipage}} & \begin{minipage}[b]{\linewidth}\centering
T4.1
\end{minipage} & \begin{minipage}[b]{\linewidth}\raggedright
Initialise the Vue project and routing
\end{minipage} & \begin{minipage}[b]{\linewidth}\centering
4
\end{minipage} \\
& & \begin{minipage}[b]{\linewidth}\centering
T4.2
\end{minipage} & \begin{minipage}[b]{\linewidth}\raggedright
Create forms
\end{minipage} & \begin{minipage}[b]{\linewidth}\centering
5
\end{minipage} \\
& & \begin{minipage}[b]{\linewidth}\centering
T4.3
\end{minipage} & \begin{minipage}[b]{\linewidth}\raggedright
Style UI and validate fields
\end{minipage} & \begin{minipage}[b]{\linewidth}\centering
3
\end{minipage} \\
& & \begin{minipage}[b]{\linewidth}\centering
T4.4
\end{minipage} & \begin{minipage}[b]{\linewidth}\raggedright
Connect the frontend with the backend Auth

API
\end{minipage} & \begin{minipage}[b]{\linewidth}\centering
2
\end{minipage} \\
\multirow{4}{=}{\begin{minipage}[b]{\linewidth}\centering
US005
\end{minipage}} &
\multirow{4}{=}{\begin{minipage}[b]{\linewidth}\centering
Handle authentication logic
\end{minipage}} & \begin{minipage}[b]{\linewidth}\centering
T5.1
\end{minipage} & \begin{minipage}[b]{\linewidth}\raggedright
Set up authentication middleware in Go

(JWT-based)
\end{minipage} & \begin{minipage}[b]{\linewidth}\centering
2
\end{minipage} \\
& & \begin{minipage}[b]{\linewidth}\centering
T5.2
\end{minipage} & \begin{minipage}[b]{\linewidth}\raggedright
Create login/signup API endpoints
\end{minipage} & \begin{minipage}[b]{\linewidth}\centering
2
\end{minipage} \\
& & \begin{minipage}[b]{\linewidth}\centering
T5.3
\end{minipage} & \begin{minipage}[b]{\linewidth}\raggedright
Secure routes using middleware
\end{minipage} & \begin{minipage}[b]{\linewidth}\centering
3
\end{minipage} \\
& & \begin{minipage}[b]{\linewidth}\centering
T5.4
\end{minipage} & \begin{minipage}[b]{\linewidth}\raggedright
Test authentication flow (manual +

Postman)
\end{minipage} & \begin{minipage}[b]{\linewidth}\centering
2
\end{minipage} \\
\multirow{3}{=}{\begin{minipage}[b]{\linewidth}\centering
US006
\end{minipage}} &
\multirow{3}{=}{\begin{minipage}[b]{\linewidth}\centering
RBAC (Role-Based Access Control)
\end{minipage}} & \begin{minipage}[b]{\linewidth}\centering
T6.1
\end{minipage} & \begin{minipage}[b]{\linewidth}\raggedright
Define roles: admin, developer, viewer
\end{minipage} & \begin{minipage}[b]{\linewidth}\centering
1
\end{minipage} \\
& & \begin{minipage}[b]{\linewidth}\centering
T6.2
\end{minipage} & \begin{minipage}[b]{\linewidth}\raggedright
Add RBAC checks to middleware
\end{minipage} & \begin{minipage}[b]{\linewidth}\centering
3
\end{minipage} \\
& & \begin{minipage}[b]{\linewidth}\centering
T6.3
\end{minipage} & \begin{minipage}[b]{\linewidth}\raggedright
Test access restrictions based on role
\end{minipage} & \begin{minipage}[b]{\linewidth}\centering
1
\end{minipage} \\
\multirow{3}{=}{\begin{minipage}[b]{\linewidth}\centering
US007
\end{minipage}} &
\multirow{3}{=}{\begin{minipage}[b]{\linewidth}\centering
Implement Authentication tools
\end{minipage}} & \begin{minipage}[b]{\linewidth}\centering
T7.1
\end{minipage} & \begin{minipage}[b]{\linewidth}\raggedright
Implement password hashing

(e.g., bcrypt)
\end{minipage} & \begin{minipage}[b]{\linewidth}\centering
1
\end{minipage} \\
& & \begin{minipage}[b]{\linewidth}\centering
T7.2
\end{minipage} & \begin{minipage}[b]{\linewidth}\raggedright
Add an email format validator and strong

password policy
\end{minipage} & \begin{minipage}[b]{\linewidth}\centering
1
\end{minipage} \\
& & \begin{minipage}[b]{\linewidth}\centering
T7.3
\end{minipage} & \begin{minipage}[b]{\linewidth}\raggedright
Write unit tests for the auth logic
\end{minipage} & \begin{minipage}[b]{\linewidth}\centering
3
\end{minipage} \\
\midrule\noalign{}
\endhead
\bottomrule\noalign{}
\endlastfoot
\end{longtable}
}

\begin{enumerate}
\def\labelenumi{\arabic{enumi}.}
\setcounter{enumi}{1}
\item
  Sprint 2:
\end{enumerate}

{\def\LTcaptype{} % do not increment counter
\begin{longtable}[]{@{}
  >{\raggedright\arraybackslash}p{(\linewidth - 8\tabcolsep) * \real{0.0921}}
  >{\raggedright\arraybackslash}p{(\linewidth - 8\tabcolsep) * \real{0.3048}}
  >{\raggedright\arraybackslash}p{(\linewidth - 8\tabcolsep) * \real{0.1058}}
  >{\raggedright\arraybackslash}p{(\linewidth - 8\tabcolsep) * \real{0.3915}}
  >{\raggedright\arraybackslash}p{(\linewidth - 8\tabcolsep) * \real{0.1058}}@{}}
\toprule\noalign{}
\begin{minipage}[b]{\linewidth}\raggedright
\end{minipage} & \begin{minipage}[b]{\linewidth}\raggedright
\end{minipage} & \begin{minipage}[b]{\linewidth}\raggedright
\end{minipage} & \begin{minipage}[b]{\linewidth}\raggedright
\end{minipage} & \begin{minipage}[b]{\linewidth}\raggedright
\end{minipage} \\
\begin{minipage}[b]{\linewidth}\centering
UserStoryId
\end{minipage} & \begin{minipage}[b]{\linewidth}\centering
UserStory
\end{minipage} & \begin{minipage}[b]{\linewidth}\centering
TaskId
\end{minipage} & \begin{minipage}[b]{\linewidth}\raggedright
Task
\end{minipage} & \begin{minipage}[b]{\linewidth}\centering
Est. effort
\end{minipage} \\
\multirow{4}{=}{\begin{minipage}[b]{\linewidth}\centering
US008
\end{minipage}} &
\multirow{4}{=}{\begin{minipage}[b]{\linewidth}\centering
Implement dashboard metrics UI
\end{minipage}} & \begin{minipage}[b]{\linewidth}\centering
T8.1
\end{minipage} & \begin{minipage}[b]{\linewidth}\raggedright
Design wireframe for metrics layout
\end{minipage} & \begin{minipage}[b]{\linewidth}\raggedleft
1
\end{minipage} \\
& & \begin{minipage}[b]{\linewidth}\centering
T8.2
\end{minipage} & \begin{minipage}[b]{\linewidth}\raggedright
Create Vue.js components for KPI cards
\end{minipage} & \begin{minipage}[b]{\linewidth}\raggedleft
2
\end{minipage} \\
& & \begin{minipage}[b]{\linewidth}\centering
T8.3
\end{minipage} & \begin{minipage}[b]{\linewidth}\raggedright
Integrate static data for testing UI rendering
\end{minipage} & \begin{minipage}[b]{\linewidth}\raggedleft
2
\end{minipage} \\
& & \begin{minipage}[b]{\linewidth}\centering
T8.4
\end{minipage} & \begin{minipage}[b]{\linewidth}\raggedright
Setup responsive layout with basic CSS/utility
\end{minipage} & \begin{minipage}[b]{\linewidth}\raggedleft
1
\end{minipage} \\
\multirow{4}{=}{\begin{minipage}[b]{\linewidth}\centering
US009
\end{minipage}} &
\multirow{4}{=}{\begin{minipage}[b]{\linewidth}\centering
Develop neccessary API\textquotesingle s
\end{minipage}} & \begin{minipage}[b]{\linewidth}\centering
T9.1
\end{minipage} & \begin{minipage}[b]{\linewidth}\raggedright
Define API endpoints for dashboard metrics (OpenAPI)
\end{minipage} & \begin{minipage}[b]{\linewidth}\raggedleft
3
\end{minipage} \\
& & \begin{minipage}[b]{\linewidth}\centering
T9.2
\end{minipage} & \begin{minipage}[b]{\linewidth}\raggedright
Implement GET endpoints (e.g., /metrics, /summary)
\end{minipage} & \begin{minipage}[b]{\linewidth}\raggedleft
1
\end{minipage} \\
& & \begin{minipage}[b]{\linewidth}\centering
T9.3
\end{minipage} & \begin{minipage}[b]{\linewidth}\raggedright
Connect to database to fetch live data
\end{minipage} & \begin{minipage}[b]{\linewidth}\raggedleft
1
\end{minipage} \\
& & \begin{minipage}[b]{\linewidth}\centering
T9.4
\end{minipage} & \begin{minipage}[b]{\linewidth}\raggedright
Add error handling and logging
\end{minipage} & \begin{minipage}[b]{\linewidth}\raggedleft
2
\end{minipage} \\
\multirow{4}{=}{\begin{minipage}[b]{\linewidth}\centering
US010
\end{minipage}} &
\multirow{4}{=}{\begin{minipage}[b]{\linewidth}\centering
Implement Errors /Logs UI
\end{minipage}} & \begin{minipage}[b]{\linewidth}\centering
T10.1
\end{minipage} & \begin{minipage}[b]{\linewidth}\raggedright
Design UI layout for error/logs panel
\end{minipage} & \begin{minipage}[b]{\linewidth}\raggedleft
1
\end{minipage} \\
& & \begin{minipage}[b]{\linewidth}\centering
T10.2
\end{minipage} & \begin{minipage}[b]{\linewidth}\raggedright
Build Vue components for logs table + error viewer
\end{minipage} & \begin{minipage}[b]{\linewidth}\raggedleft
2
\end{minipage} \\
& & \begin{minipage}[b]{\linewidth}\centering
T10.3
\end{minipage} & \begin{minipage}[b]{\linewidth}\raggedright
Add pagination, filters (by service, level, timestamp, etc.)
\end{minipage} & \begin{minipage}[b]{\linewidth}\raggedleft
1
\end{minipage} \\
& & \begin{minipage}[b]{\linewidth}\centering
T10.4
\end{minipage} & \begin{minipage}[b]{\linewidth}\raggedright
Connect frontend to API
\end{minipage} & \begin{minipage}[b]{\linewidth}\raggedleft
1
\end{minipage} \\
\multirow{4}{=}{\begin{minipage}[b]{\linewidth}\centering
US011
\end{minipage}} &
\multirow{4}{=}{\begin{minipage}[b]{\linewidth}\centering
Integrate Dashboard with Backend Services
\end{minipage}} & \begin{minipage}[b]{\linewidth}\centering
T11.1
\end{minipage} & \begin{minipage}[b]{\linewidth}\raggedright
Connect frontend dashboard with real-time APIs
\end{minipage} & \begin{minipage}[b]{\linewidth}\raggedleft
2
\end{minipage} \\
& & \begin{minipage}[b]{\linewidth}\centering
T11.2
\end{minipage} & \begin{minipage}[b]{\linewidth}\raggedright
Implement periodic refresh of metrics/logs
\end{minipage} & \begin{minipage}[b]{\linewidth}\raggedleft
2
\end{minipage} \\
& & \begin{minipage}[b]{\linewidth}\centering
T11.3
\end{minipage} & \begin{minipage}[b]{\linewidth}\raggedright
Integrate WebSocket/RMQ or polling mechanism
\end{minipage} & \begin{minipage}[b]{\linewidth}\raggedleft
5
\end{minipage} \\
& & \begin{minipage}[b]{\linewidth}\centering
T11.4
\end{minipage} & \begin{minipage}[b]{\linewidth}\raggedright
Test and debug data sync between backend and UI
\end{minipage} & \begin{minipage}[b]{\linewidth}\raggedleft
2
\end{minipage} \\
\multirow{4}{=}{\begin{minipage}[b]{\linewidth}\centering
US012
\end{minipage}} &
\multirow{4}{=}{\begin{minipage}[b]{\linewidth}\centering
Handle Dashboard Active Services
\end{minipage}} & \begin{minipage}[b]{\linewidth}\centering
T12.1
\end{minipage} & \begin{minipage}[b]{\linewidth}\raggedright
Build API to track active services (status, heartbeat)
\end{minipage} & \begin{minipage}[b]{\linewidth}\raggedleft
3
\end{minipage} \\
& & \begin{minipage}[b]{\linewidth}\centering
T12.2
\end{minipage} & \begin{minipage}[b]{\linewidth}\raggedright
Create UI indicators for live/inactive service states
\end{minipage} & \begin{minipage}[b]{\linewidth}\raggedleft
2
\end{minipage} \\
& & \begin{minipage}[b]{\linewidth}\centering
T12.3
\end{minipage} & \begin{minipage}[b]{\linewidth}\raggedright
Sync service status with backend in real-time
\end{minipage} & \begin{minipage}[b]{\linewidth}\raggedleft
1
\end{minipage} \\
& & \begin{minipage}[b]{\linewidth}\centering
T12.4
\end{minipage} & \begin{minipage}[b]{\linewidth}\raggedright
Add error handling for disconnected/missing services
\end{minipage} & \begin{minipage}[b]{\linewidth}\raggedleft
2
\end{minipage} \\
\midrule\noalign{}
\endhead
\bottomrule\noalign{}
\endlastfoot
\end{longtable}
}

\begin{enumerate}
\def\labelenumi{\arabic{enumi}.}
\setcounter{enumi}{2}
\item
  Sprint3:
\end{enumerate}

{\def\LTcaptype{} % do not increment counter
\begin{longtable}[]{@{}
  >{\raggedright\arraybackslash}p{(\linewidth - 8\tabcolsep) * \real{0.1127}}
  >{\raggedright\arraybackslash}p{(\linewidth - 8\tabcolsep) * \real{0.2188}}
  >{\raggedright\arraybackslash}p{(\linewidth - 8\tabcolsep) * \real{0.0902}}
  >{\raggedright\arraybackslash}p{(\linewidth - 8\tabcolsep) * \real{0.4761}}
  >{\raggedright\arraybackslash}p{(\linewidth - 8\tabcolsep) * \real{0.1021}}@{}}
\toprule\noalign{}
\begin{minipage}[b]{\linewidth}\centering
UserStoryId
\end{minipage} & \begin{minipage}[b]{\linewidth}\centering
UserStory
\end{minipage} & \begin{minipage}[b]{\linewidth}\centering
TaskId
\end{minipage} & \begin{minipage}[b]{\linewidth}\raggedright
Task
\end{minipage} & \begin{minipage}[b]{\linewidth}\centering
Est. effort
\end{minipage} \\
\multirow{7}{=}{\begin{minipage}[b]{\linewidth}\centering
US013
\end{minipage}} &
\multirow{7}{=}{\begin{minipage}[b]{\linewidth}\centering
Implement Dashboard

Options of Pipeline
\end{minipage}} & \begin{minipage}[b]{\linewidth}\centering
T13.1
\end{minipage} & \begin{minipage}[b]{\linewidth}\raggedright
Define UI/UX requirements for pipeline dashboard

(wireframes, user flow)
\end{minipage} & \begin{minipage}[b]{\linewidth}\centering
4
\end{minipage} \\
& & \begin{minipage}[b]{\linewidth}\centering
T13.2
\end{minipage} & \begin{minipage}[b]{\linewidth}\raggedright
Set up frontend components for pipeline visualization

(jobs, stages, status)
\end{minipage} & \begin{minipage}[b]{\linewidth}\centering
2
\end{minipage} \\
& & \begin{minipage}[b]{\linewidth}\centering
T13.3
\end{minipage} & \begin{minipage}[b]{\linewidth}\raggedright
Implement backend endpoints to fetch pipeline data

(status, logs, history).
\end{minipage} & \begin{minipage}[b]{\linewidth}\centering
4
\end{minipage} \\
& & \begin{minipage}[b]{\linewidth}\centering
T13.4
\end{minipage} & \begin{minipage}[b]{\linewidth}\raggedright
Integrate real-time updates (WebSockets or polling).
\end{minipage} & \begin{minipage}[b]{\linewidth}\centering
3
\end{minipage} \\
& & \begin{minipage}[b]{\linewidth}\centering
T13.5
\end{minipage} & \begin{minipage}[b]{\linewidth}\raggedright
Add role-based access for dashboard visibility.
\end{minipage} & \begin{minipage}[b]{\linewidth}\centering
4
\end{minipage} \\
& & \begin{minipage}[b]{\linewidth}\centering
T13.6
\end{minipage} & \begin{minipage}[b]{\linewidth}\raggedright
Testing \& validation of dashboard metrics.
\end{minipage} & \begin{minipage}[b]{\linewidth}\centering
3
\end{minipage} \\
& & \begin{minipage}[b]{\linewidth}\centering
T13.7
\end{minipage} & \begin{minipage}[b]{\linewidth}\raggedright
Documentation of usage and configuration.
\end{minipage} & \begin{minipage}[b]{\linewidth}\centering
2
\end{minipage} \\
\multirow{8}{=}{\begin{minipage}[b]{\linewidth}\centering
US014
\end{minipage}} &
\multirow{8}{=}{\begin{minipage}[b]{\linewidth}\centering
Develop Necessary APIs
\end{minipage}} & \begin{minipage}[b]{\linewidth}\centering
T14.1
\end{minipage} & \begin{minipage}[b]{\linewidth}\raggedright
Define API contract (OpenAPI/Swagger specification)
\end{minipage} & \begin{minipage}[b]{\linewidth}\centering
2
\end{minipage} \\
& & \begin{minipage}[b]{\linewidth}\centering
T14.2
\end{minipage} & \begin{minipage}[b]{\linewidth}\raggedright
Implement API for pipeline creation
\end{minipage} & \begin{minipage}[b]{\linewidth}\centering
4
\end{minipage} \\
& & \begin{minipage}[b]{\linewidth}\centering
T14.3
\end{minipage} & \begin{minipage}[b]{\linewidth}\raggedright
Implement API for pipeline execution trigger
\end{minipage} & \begin{minipage}[b]{\linewidth}\centering
4
\end{minipage} \\
& & \begin{minipage}[b]{\linewidth}\centering
T14.4
\end{minipage} & \begin{minipage}[b]{\linewidth}\raggedright
Implement API for pipeline status retrieval.
\end{minipage} & \begin{minipage}[b]{\linewidth}\centering
4
\end{minipage} \\
& & \begin{minipage}[b]{\linewidth}\centering
T14.5
\end{minipage} & \begin{minipage}[b]{\linewidth}\raggedright
Implement API for pipeline history retrieval
\end{minipage} & \begin{minipage}[b]{\linewidth}\centering
4
\end{minipage} \\
& & \begin{minipage}[b]{\linewidth}\centering
T14.6
\end{minipage} & \begin{minipage}[b]{\linewidth}\raggedright
Apply authentication \& authorization middleware.
\end{minipage} & \begin{minipage}[b]{\linewidth}\centering
4
\end{minipage} \\
& & \begin{minipage}[b]{\linewidth}\centering
T14.7
\end{minipage} & \begin{minipage}[b]{\linewidth}\raggedright
Unit testing of API endpoints
\end{minipage} & \begin{minipage}[b]{\linewidth}\centering
4
\end{minipage} \\
& & \begin{minipage}[b]{\linewidth}\centering
T14.8
\end{minipage} & \begin{minipage}[b]{\linewidth}\raggedright
Write API documentation.
\end{minipage} & \begin{minipage}[b]{\linewidth}\centering
3
\end{minipage} \\
\multirow{8}{=}{\begin{minipage}[b]{\linewidth}\centering
US015
\end{minipage}} &
\multirow{8}{=}{\begin{minipage}[b]{\linewidth}\centering
Implement Pipeline Tools
\end{minipage}} & \begin{minipage}[b]{\linewidth}\centering
T15.1
\end{minipage} & \begin{minipage}[b]{\linewidth}\raggedright
Evaluate and choose CI/CD tools (Jenkins,

GitHub Actions, GitLab CI, etc.).
\end{minipage} & \begin{minipage}[b]{\linewidth}\centering
4
\end{minipage} \\
& & \begin{minipage}[b]{\linewidth}\centering
T15.2
\end{minipage} & \begin{minipage}[b]{\linewidth}\raggedright
Configure chosen CI/CD tool with project repository
\end{minipage} & \begin{minipage}[b]{\linewidth}\centering
4
\end{minipage} \\
& & \begin{minipage}[b]{\linewidth}\centering
T15.3
\end{minipage} & \begin{minipage}[b]{\linewidth}\raggedright
Define pipeline stages (build, test, deploy)
\end{minipage} & \begin{minipage}[b]{\linewidth}\centering
3
\end{minipage} \\
& & \begin{minipage}[b]{\linewidth}\centering
T15.4
\end{minipage} & \begin{minipage}[b]{\linewidth}\raggedright
Integrate automated testing into pipeline
\end{minipage} & \begin{minipage}[b]{\linewidth}\centering
3
\end{minipage} \\
& & \begin{minipage}[b]{\linewidth}\centering
T15.5
\end{minipage} & \begin{minipage}[b]{\linewidth}\raggedright
Integrate build artifact storage (e.g., Docker registry)
\end{minipage} & \begin{minipage}[b]{\linewidth}\centering
4
\end{minipage} \\
& & \begin{minipage}[b]{\linewidth}\centering
T15.6
\end{minipage} & \begin{minipage}[b]{\linewidth}\raggedright
Set up environment variables \& secrets securely
\end{minipage} & \begin{minipage}[b]{\linewidth}\centering
3
\end{minipage} \\
& & \begin{minipage}[b]{\linewidth}\centering
T15.7
\end{minipage} & \begin{minipage}[b]{\linewidth}\raggedright
Write deployment scripts.
\end{minipage} & \begin{minipage}[b]{\linewidth}\centering
3
\end{minipage} \\
& & \begin{minipage}[b]{\linewidth}\centering
T15.8
\end{minipage} & \begin{minipage}[b]{\linewidth}\raggedright
Test end-to-end pipeline execution.
\end{minipage} & \begin{minipage}[b]{\linewidth}\centering
3
\end{minipage} \\
\multirow{7}{=}{\begin{minipage}[b]{\linewidth}\centering
US016
\end{minipage}} &
\multirow{7}{=}{\begin{minipage}[b]{\linewidth}\centering
Handle Pipeline Active

Actions
\end{minipage}} & \begin{minipage}[b]{\linewidth}\centering
T16.1
\end{minipage} & \begin{minipage}[b]{\linewidth}\raggedright
Implement ability to start/stop/restart pipelines from

dashboard.
\end{minipage} & \begin{minipage}[b]{\linewidth}\centering
4
\end{minipage} \\
& & \begin{minipage}[b]{\linewidth}\centering
T16.2
\end{minipage} & \begin{minipage}[b]{\linewidth}\raggedright
Add rollback functionality for failed pipelines
\end{minipage} & \begin{minipage}[b]{\linewidth}\centering
4
\end{minipage} \\
& & \begin{minipage}[b]{\linewidth}\centering
T16.3
\end{minipage} & \begin{minipage}[b]{\linewidth}\raggedright
Add logs viewer (streaming logs for running pipeline)
\end{minipage} & \begin{minipage}[b]{\linewidth}\centering
4
\end{minipage} \\
& & \begin{minipage}[b]{\linewidth}\centering
T16.4
\end{minipage} & \begin{minipage}[b]{\linewidth}\raggedright
Handle error reporting \& notifications

(email/Slack/webhook).
\end{minipage} & \begin{minipage}[b]{\linewidth}\centering
4
\end{minipage} \\
& & \begin{minipage}[b]{\linewidth}\centering
T16.5
\end{minipage} & \begin{minipage}[b]{\linewidth}\raggedright
Audit trail for pipeline actions (who did what, when).
\end{minipage} & \begin{minipage}[b]{\linewidth}\centering
3
\end{minipage} \\
& & \begin{minipage}[b]{\linewidth}\centering
T16.6
\end{minipage} & \begin{minipage}[b]{\linewidth}\raggedright
Testing of active actions on different scenarios.
\end{minipage} & \begin{minipage}[b]{\linewidth}\centering
3
\end{minipage} \\
& & \begin{minipage}[b]{\linewidth}\centering
T16.7
\end{minipage} & \begin{minipage}[b]{\linewidth}\raggedright
Document troubleshooting guidelines
\end{minipage} & \begin{minipage}[b]{\linewidth}\centering
3
\end{minipage} \\
\midrule\noalign{}
\endhead
\bottomrule\noalign{}
\endlastfoot
\end{longtable}
}

\begin{enumerate}
\def\labelenumi{\arabic{enumi}.}
\setcounter{enumi}{3}
\item
  Sprint4:
\end{enumerate}

{\def\LTcaptype{} % do not increment counter
\begin{longtable}[]{@{}
  >{\raggedright\arraybackslash}p{(\linewidth - 8\tabcolsep) * \real{0.1353}}
  >{\raggedright\arraybackslash}p{(\linewidth - 8\tabcolsep) * \real{0.2582}}
  >{\raggedright\arraybackslash}p{(\linewidth - 8\tabcolsep) * \real{0.0800}}
  >{\raggedright\arraybackslash}p{(\linewidth - 8\tabcolsep) * \real{0.4149}}
  >{\raggedright\arraybackslash}p{(\linewidth - 8\tabcolsep) * \real{0.1116}}@{}}
\toprule\noalign{}
\begin{minipage}[b]{\linewidth}\centering
UserStoryId
\end{minipage} & \begin{minipage}[b]{\linewidth}\centering
UserStory
\end{minipage} & \begin{minipage}[b]{\linewidth}\centering
TaskId
\end{minipage} & \begin{minipage}[b]{\linewidth}\centering
Task
\end{minipage} & \begin{minipage}[b]{\linewidth}\centering
Est. effort
\end{minipage} \\
\multirow{2}{=}{\begin{minipage}[b]{\linewidth}\centering
US017
\end{minipage}} &
\multirow{2}{=}{\begin{minipage}[b]{\linewidth}\centering
Implment An AI model
\end{minipage}} & \begin{minipage}[b]{\linewidth}\centering
T17.1
\end{minipage} & \begin{minipage}[b]{\linewidth}\raggedright
Integrate DeepSeek API
\end{minipage} & \begin{minipage}[b]{\linewidth}\centering
2
\end{minipage} \\
& & \begin{minipage}[b]{\linewidth}\centering
T17.2
\end{minipage} & \begin{minipage}[b]{\linewidth}\raggedright
Integrate retrun model inference into the backend
\end{minipage} & \begin{minipage}[b]{\linewidth}\centering
4
\end{minipage} \\
\multirow{2}{=}{\begin{minipage}[b]{\linewidth}\centering
US018
\end{minipage}} &
\multirow{2}{=}{\begin{minipage}[b]{\linewidth}\centering
Tag Errors with suggests fixes
\end{minipage}} & \begin{minipage}[b]{\linewidth}\centering
T18.1
\end{minipage} & \begin{minipage}[b]{\linewidth}\raggedright
Implement a tagging system to link errors with potential fixes.
\end{minipage} & \begin{minipage}[b]{\linewidth}\centering
4
\end{minipage} \\
& & \begin{minipage}[b]{\linewidth}\centering
T18.2
\end{minipage} & \begin{minipage}[b]{\linewidth}\raggedright
Provide structured metadata for developers.
\end{minipage} & \begin{minipage}[b]{\linewidth}\centering
4
\end{minipage} \\
\multirow{2}{=}{\begin{minipage}[b]{\linewidth}\centering
US019
\end{minipage}} &
\multirow{2}{=}{\begin{minipage}[b]{\linewidth}\centering
Implement automated code fixes
\end{minipage}} & \begin{minipage}[b]{\linewidth}\centering
T19.1
\end{minipage} & \begin{minipage}[b]{\linewidth}\raggedright
Build a mechanism to apply code fixes automatically.
\end{minipage} & \begin{minipage}[b]{\linewidth}\centering
4
\end{minipage} \\
& & \begin{minipage}[b]{\linewidth}\centering
T19.2
\end{minipage} & \begin{minipage}[b]{\linewidth}\raggedright
Ensure rollback or safe-guarding strategy in case of incorrect fixes
\end{minipage} & \begin{minipage}[b]{\linewidth}\centering
2
\end{minipage} \\
\multirow{2}{=}{\begin{minipage}[b]{\linewidth}\centering
US020
\end{minipage}} &
\multirow{2}{=}{\begin{minipage}[b]{\linewidth}\centering
Generate unit tests for proposed fixes
\end{minipage}} & \begin{minipage}[b]{\linewidth}\centering
T20.1
\end{minipage} & \begin{minipage}[b]{\linewidth}\raggedright
Automatically generate unit tests for every proposed code fix.
\end{minipage} & \begin{minipage}[b]{\linewidth}\centering
2
\end{minipage} \\
& & \begin{minipage}[b]{\linewidth}\centering
T20.2
\end{minipage} & \begin{minipage}[b]{\linewidth}\raggedright
Validate fixes against the existing test suite.
\end{minipage} & \begin{minipage}[b]{\linewidth}\centering
2
\end{minipage} \\
\multirow{2}{=}{\begin{minipage}[b]{\linewidth}\centering
US021
\end{minipage}} &
\multirow{2}{=}{\begin{minipage}[b]{\linewidth}\centering
Generate history of crash list
\end{minipage}} & \begin{minipage}[b]{\linewidth}\centering
T21.1
\end{minipage} & \begin{minipage}[b]{\linewidth}\raggedright
Record and archive all crash/error events.
\end{minipage} & \begin{minipage}[b]{\linewidth}\centering
3
\end{minipage} \\
& & \begin{minipage}[b]{\linewidth}\centering
T21.2
\end{minipage} & \begin{minipage}[b]{\linewidth}\raggedright
Provide a searchable history for analysis and audits.
\end{minipage} & \begin{minipage}[b]{\linewidth}\centering
2
\end{minipage} \\
\midrule\noalign{}
\endhead
\bottomrule\noalign{}
\endlastfoot
\end{longtable}
}

\begin{enumerate}
\def\labelenumi{\arabic{enumi}.}
\setcounter{enumi}{4}
\item
  Sprint 5:
\end{enumerate}

{\def\LTcaptype{} % do not increment counter
\begin{longtable}[]{@{}
  >{\raggedright\arraybackslash}p{(\linewidth - 8\tabcolsep) * \real{0.1022}}
  >{\raggedright\arraybackslash}p{(\linewidth - 8\tabcolsep) * \real{0.3167}}
  >{\raggedright\arraybackslash}p{(\linewidth - 8\tabcolsep) * \real{0.0942}}
  >{\raggedright\arraybackslash}p{(\linewidth - 8\tabcolsep) * \real{0.3734}}
  >{\raggedright\arraybackslash}p{(\linewidth - 8\tabcolsep) * \real{0.1135}}@{}}
\toprule\noalign{}
\begin{minipage}[b]{\linewidth}\centering
UserStoryId
\end{minipage} & \begin{minipage}[b]{\linewidth}\centering
UserStory
\end{minipage} & \begin{minipage}[b]{\linewidth}\centering
TaskId
\end{minipage} & \begin{minipage}[b]{\linewidth}\centering
Task
\end{minipage} & \begin{minipage}[b]{\linewidth}\centering
Est. effort
\end{minipage} \\
\multirow{6}{=}{\begin{minipage}[b]{\linewidth}\centering
US022
\end{minipage}} &
\multirow{6}{=}{\begin{minipage}[b]{\linewidth}\centering
Configure Tools integrations
\end{minipage}} & \begin{minipage}[b]{\linewidth}\centering
T22.1
\end{minipage} & \begin{minipage}[b]{\linewidth}\raggedright
Analyze required monitoring/alerting tools (PagerDuty, Slack, MS Teams,
etc.)
\end{minipage} & \begin{minipage}[b]{\linewidth}\centering
2
\end{minipage} \\
& & \begin{minipage}[b]{\linewidth}\centering
T22.2
\end{minipage} & \begin{minipage}[b]{\linewidth}\raggedright
Set up API keys and authentication for selected tools
\end{minipage} & \begin{minipage}[b]{\linewidth}\centering
1
\end{minipage} \\
& & \begin{minipage}[b]{\linewidth}\centering
T22.3
\end{minipage} & \begin{minipage}[b]{\linewidth}\raggedright
Implement integration adapters (e.g., webhook handlers)
\end{minipage} & \begin{minipage}[b]{\linewidth}\centering
1
\end{minipage} \\
& & \begin{minipage}[b]{\linewidth}\centering
T22.4
\end{minipage} & \begin{minipage}[b]{\linewidth}\raggedright
Configure environment variables for integration endpoints
\end{minipage} & \begin{minipage}[b]{\linewidth}\centering
2
\end{minipage} \\
& & \begin{minipage}[b]{\linewidth}\centering
T22.5
\end{minipage} & \begin{minipage}[b]{\linewidth}\raggedright
Test sending a sample alert to each tool
\end{minipage} & \begin{minipage}[b]{\linewidth}\centering
2
\end{minipage} \\
& & \begin{minipage}[b]{\linewidth}\centering
T22.6
\end{minipage} & \begin{minipage}[b]{\linewidth}\raggedright
Document integration setup
\end{minipage} & \begin{minipage}[b]{\linewidth}\centering
2
\end{minipage} \\
\multirow{6}{=}{\begin{minipage}[b]{\linewidth}\centering
US023
\end{minipage}} &
\multirow{6}{=}{\begin{minipage}[b]{\linewidth}\centering
Integrations notifications System
\end{minipage}} & \begin{minipage}[b]{\linewidth}\centering
T23.1
\end{minipage} & \begin{minipage}[b]{\linewidth}\raggedright
Design notification service architecture (centralized dispatcher)
\end{minipage} & \begin{minipage}[b]{\linewidth}\centering
1
\end{minipage} \\
& & \begin{minipage}[b]{\linewidth}\centering
T23.2
\end{minipage} & \begin{minipage}[b]{\linewidth}\raggedright
Implement notification templates (success, warning, error)
\end{minipage} & \begin{minipage}[b]{\linewidth}\centering
2
\end{minipage} \\
& & \begin{minipage}[b]{\linewidth}\centering
T23.3
\end{minipage} & \begin{minipage}[b]{\linewidth}\raggedright
Add routing logic (choose tool/channel based on rules)
\end{minipage} & \begin{minipage}[b]{\linewidth}\centering
2
\end{minipage} \\
& & \begin{minipage}[b]{\linewidth}\centering
T23.4
\end{minipage} & \begin{minipage}[b]{\linewidth}\raggedright
Build retry mechanism for failed notifications
\end{minipage} & \begin{minipage}[b]{\linewidth}\centering
1
\end{minipage} \\
& & \begin{minipage}[b]{\linewidth}\centering
T23.5
\end{minipage} & \begin{minipage}[b]{\linewidth}\raggedright
Unit test notification service
\end{minipage} & \begin{minipage}[b]{\linewidth}\centering
1
\end{minipage} \\
& & \begin{minipage}[b]{\linewidth}\centering
T23.6
\end{minipage} & \begin{minipage}[b]{\linewidth}\raggedright
Verify end-to-end delivery to integrated tools
\end{minipage} & \begin{minipage}[b]{\linewidth}\centering
1
\end{minipage} \\
\multirow{5}{=}{\begin{minipage}[b]{\linewidth}\centering
US024
\end{minipage}} &
\multirow{5}{=}{\begin{minipage}[b]{\linewidth}\centering
Implement Billing Alerts
\end{minipage}} & \begin{minipage}[b]{\linewidth}\centering
T24.1
\end{minipage} & \begin{minipage}[b]{\linewidth}\raggedright
Identify billing thresholds (quota, over-usage, abnormal costs)
\end{minipage} & \begin{minipage}[b]{\linewidth}\centering
2
\end{minipage} \\
& & \begin{minipage}[b]{\linewidth}\centering
T24.2
\end{minipage} & \begin{minipage}[b]{\linewidth}\raggedright
Implement rule engine for billing alerts
\end{minipage} & \begin{minipage}[b]{\linewidth}\centering
1
\end{minipage} \\
& & \begin{minipage}[b]{\linewidth}\centering
T24.3
\end{minipage} & \begin{minipage}[b]{\linewidth}\raggedright
Connect billing system data to alert service
\end{minipage} & \begin{minipage}[b]{\linewidth}\centering
2
\end{minipage} \\
& & \begin{minipage}[b]{\linewidth}\centering
T24.4
\end{minipage} & \begin{minipage}[b]{\linewidth}\raggedright
Create alert templates (email/SMS/integration)
\end{minipage} & \begin{minipage}[b]{\linewidth}\centering
2
\end{minipage} \\
& & \begin{minipage}[b]{\linewidth}\centering
T24.5
\end{minipage} & \begin{minipage}[b]{\linewidth}\raggedright
Test alert triggering with simulated billing events
\end{minipage} & \begin{minipage}[b]{\linewidth}\centering
2
\end{minipage} \\
\multirow{5}{=}{\begin{minipage}[b]{\linewidth}\centering
US025
\end{minipage}} &
\multirow{5}{=}{\begin{minipage}[b]{\linewidth}\centering
Throttle Alerts
\end{minipage}} & \begin{minipage}[b]{\linewidth}\centering
T25.1
\end{minipage} & \begin{minipage}[b]{\linewidth}\raggedright
Analyze alert flood scenarios
\end{minipage} & \begin{minipage}[b]{\linewidth}\centering
2
\end{minipage} \\
& & \begin{minipage}[b]{\linewidth}\centering
T25.2
\end{minipage} & \begin{minipage}[b]{\linewidth}\raggedright
Implement throttling middleware in alert pipeline
\end{minipage} & \begin{minipage}[b]{\linewidth}\centering
3
\end{minipage} \\
& & \begin{minipage}[b]{\linewidth}\centering
T25.3
\end{minipage} & \begin{minipage}[b]{\linewidth}\raggedright
Store last-sent timestamp per error type (Redis/DB cache)
\end{minipage} & \begin{minipage}[b]{\linewidth}\centering
2
\end{minipage} \\
& & \begin{minipage}[b]{\linewidth}\centering
T25.4
\end{minipage} & \begin{minipage}[b]{\linewidth}\raggedright
Enforce max frequency (1/min per type)
\end{minipage} & \begin{minipage}[b]{\linewidth}\centering
1
\end{minipage} \\
& & \begin{minipage}[b]{\linewidth}\centering
T25.5
\end{minipage} & \begin{minipage}[b]{\linewidth}\raggedright
Add logging for suppressed alerts
\end{minipage} & \begin{minipage}[b]{\linewidth}\centering
2
\end{minipage} \\
\multirow{6}{=}{\begin{minipage}[b]{\linewidth}\centering
US026
\end{minipage}} &
\multirow{6}{=}{\begin{minipage}[b]{\linewidth}\centering
Handle Alerting Rules
\end{minipage}} & \begin{minipage}[b]{\linewidth}\centering
T26.1
\end{minipage} & \begin{minipage}[b]{\linewidth}\raggedright
Design rules schema (conditions, thresholds, channels)
\end{minipage} & \begin{minipage}[b]{\linewidth}\centering
2
\end{minipage} \\
& & \begin{minipage}[b]{\linewidth}\centering
T26.2
\end{minipage} & \begin{minipage}[b]{\linewidth}\raggedright
Build CRUD API for alert rules (create/update/delete)
\end{minipage} & \begin{minipage}[b]{\linewidth}\centering
2
\end{minipage} \\
& & \begin{minipage}[b]{\linewidth}\centering
T26.3
\end{minipage} & \begin{minipage}[b]{\linewidth}\raggedright
Implement rules evaluation engine
\end{minipage} & \begin{minipage}[b]{\linewidth}\centering
1
\end{minipage} \\
& & \begin{minipage}[b]{\linewidth}\centering
T26.4
\end{minipage} & \begin{minipage}[b]{\linewidth}\raggedright
Store rules in DB
\end{minipage} & \begin{minipage}[b]{\linewidth}\centering
3
\end{minipage} \\
& & \begin{minipage}[b]{\linewidth}\centering
T26.5
\end{minipage} & \begin{minipage}[b]{\linewidth}\raggedright
Integrate rules engine with notification dispatcher
\end{minipage} & \begin{minipage}[b]{\linewidth}\centering
2
\end{minipage} \\
& & \begin{minipage}[b]{\linewidth}\centering
T26.4
\end{minipage} & \begin{minipage}[b]{\linewidth}\raggedright
Build UI (basic or API endpoints) to manage rules
\end{minipage} & \begin{minipage}[b]{\linewidth}\centering
4
\end{minipage} \\
\midrule\noalign{}
\endhead
\bottomrule\noalign{}
\endlastfoot
\end{longtable}
}

\begin{enumerate}
\def\labelenumi{\arabic{enumi}.}
\setcounter{enumi}{5}
\item
  Sprint 6:
\end{enumerate}  
{\def\LTcaptype{} % do not increment counter
\begin{longtable}[]{@{}
  >{\raggedright\arraybackslash}p{(\linewidth - 8\tabcolsep) * \real{0.1022}}
  >{\raggedright\arraybackslash}p{(\linewidth - 8\tabcolsep) * \real{0.3167}}
  >{\raggedright\arraybackslash}p{(\linewidth - 8\tabcolsep) * \real{0.0942}}
  >{\raggedright\arraybackslash}p{(\linewidth - 8\tabcolsep) * \real{0.3734}}
  >{\raggedright\arraybackslash}p{(\linewidth - 8\tabcolsep) * \real{0.1135}}@{}}
\toprule\noalign{}
\begin{tabular}{@{}lllll@{}}
UserStoryId & UserStory & TaskId & Task & Est. Effort \\
\midrule
US027 & Encrypt sensitive data using (AES-256) & T02701 & Analyze and identify sensitive data fields requiring encryption & 8 \\
 &  & T02702 & Implement AES-256 encryption for database fields & 12 \\
 &  & T02703 & Develop key management system for encryption keys & 10 \\
 &  & T02704 & Create data encryption/decryption utilities & 8 \\
 &  & T02705 & Test encryption implementation and performance & 7 \\
US028 & Implement GDPR-compliant audit logging & T02801 & Define GDPR audit logging requirements and data scope & 4 \\
 &  & T02802 & Design audit log schema and storage structure & 5 \\
 &  & T02803 & Implement audit logging middleware/interceptors & 8 \\
 &  & T02804 & Create log retrieval and export functionality & 6 \\
 &  & T02805 & Implement log retention and deletion policies & 5 \\
US029 & Compute the usage of system & T02901 & Define usage metrics and tracking requirements & 10 \\
 &  & T02902 & Implement usage data collection mechanisms & 12 \\
 &  & T02903 & Create usage analytics and reporting module & 15 \\
 &  & T02904 & Develop usage dashboard and visualization & 8 \\
 &  & T02905 & Implement usage alerts and notifications & 5 \\
US030 & Handle payments services & T03001 & Research and select payment gateway integration & 6 \\
 &  & T03002 & Implement payment processing workflow & 10 \\
 &  & T03003 & Create payment transaction logging and tracking & 8 \\
 &  & T03004 & Develop refund and cancellation handling & 6 \\
 &  & T03005 & Implement payment security and PCI compliance measures & 12 \\
 &  & T03006 & Test payment integration end-to-end & 8 \\
US031 & Implement role-based permissions & T03101 & Define role hierarchy and permission matrix & 8 \\
 &  & T03102 & Create database schema for roles and permissions & 6 \\
 &  & T03103 & Implement permission checking middleware & 10 \\
 &  & T03104 & Develop user-role assignment interface & 8 \\
 &  & T03105 & Create permission testing and validation suite & 6 \\
\bottomrule
\end{tabular}
\end{longtable}
\item
  Sprint 7:
\end{enumerate}
{\def\LTcaptype{} % do not increment counter
\begin{longtable}[]{@{}
  >{\raggedright\arraybackslash}p{(\linewidth - 8\tabcolsep) * \real{0.1022}}
  >{\raggedright\arraybackslash}p{(\linewidth - 8\tabcolsep) * \real{0.3167}}
  >{\raggedright\arraybackslash}p{(\linewidth - 8\tabcolsep) * \real{0.0942}}
  >{\raggedright\arraybackslash}p{(\linewidth - 8\tabcolsep) * \real{0.3734}}
  >{\raggedright\arraybackslash}p{(\linewidth - 8\tabcolsep) * \real{0.1135}}@{}}
\toprule\noalign{}
\begin{tabular}{@{}lllll@{}}
UserStoryId & UserStory & TaskId & Task & Est. Effort \\
\midrule
US032 & Create Plugin for NodeJs & T03201 & Design plugin architecture and API interface & 6 \\
 &  & T03202 & Implement core plugin functionality and methods & 10 \\
 &  & T03203 & Write unit tests and integration tests for the plugin & 8 \\
 &  & T03204 & Package and publish plugin to NPM registry & 4 \\
 &  & T03205 & Create basic usage examples and README & 4 \\
US033 & Create SDK for Flutter/Dart & T03301 & Design SDK architecture and data models (Dart Classes) & 8 \\
 &  & T03302 & Implement API client and network communication layer & 10 \\
 &  & T03303 & Develop core SDK features and methods & 10 \\
 &  & T03304 & Write comprehensive unit and widget tests & 8 \\
 &  & T03305 & Document the SDK API and publish to pub.dev & 6 \\
US034 & Handle Service Activation & T03401 & Design service activation workflow (trial, paid, etc.) & 6 \\
 &  & T03402 & Implement activation endpoint and status tracking & 8 \\
 &  & T03403 & Develop license key generation and validation logic & 8 \\
 &  & T03404 & Create admin interface for managing activations & 6 \\
 &  & T03405 & Test activation/deactivation scenarios end-to-end & 6 \\
US035 & Create Documentation for exploring Service & T03501 & Outline documentation structure and user journeys & 5 \\
 &  & T03502 & Write "Getting Started" guide and installation instructions & 6 \\
 &  & T03503 & Create comprehensive API reference documentation & 12 \\
 &  & T03504 & Develop tutorials and code samples for common use cases & 10 \\
 &  & T03505 & Set up and deploy documentation site (e.g., GitBook, Docusaurus) & 5 \\
\bottomrule
\end{tabular}
\end{longtable}
\textbf{\hfill\break
}

\end{document}
