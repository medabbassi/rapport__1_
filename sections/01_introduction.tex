% Section 1: Introduction
% This file contains the introduction, background, and project overview

\section{Introduction}

\subsection{Background and Context}

In the field of software development, error handling represents a major challenge. Bugs and anomalies, whether detected during the development phase or in production, require rapid and effective intervention to avoid service interruptions and financial losses. However, traditional approaches rely on manual detection, complex diagnosis and often time-consuming remediation, slowing down the development cycle. Additionally, the integration of DevOps and CI/CD solutions remains a manual process, requiring constant monitoring and human intervention.

\subsection{Problem Statement}

Errors and bugs are common in software development and can slow down production cycles. Manually identifying, analysing, and correcting these errors takes time, consumes resources, and can impact the quality of the final product. Additionally, DevOps integration necessitates continuous monitoring and manual interventions, which can make it challenging to deploy applications quickly and securely.

ErrorZen is an intelligent platform that automatically detects, analyses and fixes errors in backend, frontend and mobile applications. Thanks to artificial intelligence and advanced DevOps integration, ErrorZen:

\begin{itemize}
\item $\checkmark$ Identifies errors and anomalies in real time
\item $\checkmark$ Offers AI-based auto-correction solutions
\item $\checkmark$ Integrates CI/CD pipelines to automate testing and deployments
\item $\checkmark$ Notifies developers in real-time via Slack, email, or webhook
\end{itemize}

This solution helps reduce bug-fixing time, improve application quality, and speed up the development cycle.

\subsection{Objectives of the Project}

This project aims to design and develop ErrorZen, an intelligent platform for automating error management in web and mobile applications. The main objectives are:

\begin{itemize}
\item Automatic error detection in real time on different platforms (frontend, backend, mobile)
\item Automated analysis and correction of anomalies using artificial intelligence models
\item Advanced DevOps integration, enabling automation of testing and deployment after patching
\item Centralisation and visualisation of errors via an interactive dashboard, optimised with GraphQL and gRPC for effective communication
\item Instant notification to development teams via tools such as Slack, email or webhooks
\end{itemize}

\subsection{Agile Methodology: Why Scrum/RAD?}

The project will follow a RAD (Rapid Application Development) approach to ensure rapid and iterative development and will respect the Agile-Scrum development cycle. The platform will be built with:

\begin{itemize}
\item \textbf{Backend:} Go, PostgreSQL, gRPC/Rest API for communication between services
\item \textbf{Frontend:} Vue.js with GraphQL Client for integration
\item \textbf{Artificial Intelligence:} Models based on deepseek api for automatic error correction
\item \textbf{DevOps:} CI/CD via GitHub Actions/Jenkins, deployment with Docker, and Kubernetes on AWS
\end{itemize}

\subsection{Expected Results}

\begin{itemize}
\item A significant reduction in error correction time in the development cycle
\item Improved application reliability through self-correction and automated testing
\item Acceleration of the production process via DevOps automation
\item An intuitive interface allows developers to track and manage errors in real time
\end{itemize}

ErrorZen will provide an innovative solution to optimise error handling and automate DevOps cycles, thereby reducing developer workload and improving software quality.

\subsection{Overview of Sprints}

The project was divided into \textbf{7 sprints}, each lasting \textbf{two weeks}:

\begin{itemize}
\item \textbf{Sprint 1:} Project setup, requirements gathering, and initial architecture
\item \textbf{Sprint 2:} Development of core modules (e.g., user authentication, data models)
\item \textbf{Sprint 3:} Integration with back-end, user interface refinements, testing
\item \textbf{Sprint 4:} Final features, deployment, documentation, and user training
\end{itemize}

Each sprint had its backlog, planning session, and review meeting. This structure helped ensure that the development process was both flexible and measurable, with tangible results delivered at each iteration.