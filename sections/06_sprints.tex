% Section 6: Sprints Implementation

\section{Sprint Implementation}

\subsection{Sprint 1: Project Setup \& Initial Design}

\textbf{Duration:} March 25, 2025 - April 7, 2025 (2 weeks)

\textbf{Sprint Goal:} Establish project foundation, technology stack, and initial architecture.

\begin{table}[H]
\centering
\caption{Sprint 1 - Detailed Task Breakdown}
\scriptsize
\begin{tabular}{|p{1cm}|>{\raggedright\arraybackslash}p{3.5cm}|p{1cm}|>{\raggedright\arraybackslash}p{7cm}|p{1cm}|}
\hline
\textbf{Story} & \textbf{Description} & \textbf{Task} & \textbf{Task Description} & \textbf{Hours} \\ \hline
\multirow{5}{*}{US001} & \multirow{5}{*}{Set up Go/gRPC /RestfulApi \newline backend for time \newline error ingestion} & T1.1 & Initialize Go module and workspace & 3h \\
\cline{3-5}
& & T1.2 & Set up gRPC server and define proto files & 3h \\
\cline{3-5}
& & T1.3 & Create basic REST API router & 2h \\
\cline{3-5}
& & T1.4 & Add error logging middleware (e.g., interceptors, logging libs) & 5h \\
\cline{3-5}
& & T1.5 & Test local gRPC and REST endpoints using Postman and grpcurl & 3h \\ \hline
\multirow{5}{*}{US002} & \multirow{5}{*}{Implement PostgresSQL for \newline structured error storage \newline with WAL(Write-Ahead logging)} & T2.1 & Install and configure PostgreSQL locally & 1h \\
\cline{3-5}
& & T2.2 & Design initial schema for error logs (e.g., errors, services, projects) & 5h \\
\cline{3-5}
& & T2.3 & Configure Write-Ahead Logging (WAL) for safe/error-tolerant write operations & 2h \\
\cline{3-5}
& & T2.4 & Create migration script for schema initialization using Go & 4h \\
\cline{3-5}
& & T2.5 & Write DB connection logic in Go with retry and health-check capabilities & 2h \\ \hline
\multirow{3}{*}{US003} & \multirow{3}{*}{Design RestApi for \newline external integration} & T3.1 & Define OpenAPI spec (Swagger) for public-facing endpoints & 4h \\
\cline{3-5}
& & T3.2 & Implement one sample REST endpoint & 1h \\
\cline{3-5}
& & T3.3 & Add basic input validation and error handling & 2h \\ \hline
\multirow{4}{*}{US004} & \multirow{4}{*}{Implement Authentification \newline UI with VueJs} & T4.1 & Initialize Vue project and routing & 4h \\
\cline{3-5}
& & T4.2 & Create forms & 5h \\
\cline{3-5}
& & T4.3 & Style UI and validate fields & 3h \\
\cline{3-5}
& & T4.4 & Connect frontend with backend Auth API & 2h \\ \hline
\multirow{4}{*}{US005} & \multirow{4}{*}{Handle authentication \newline logic} & T5.1 & Set up authentication middleware in Go (JWT-based) & 2h \\
\cline{3-5}
& & T5.2 & Create login/signup API endpoints & 2h \\
\cline{3-5}
& & T5.3 & Secure routes using middleware & 3h \\
\cline{3-5}
& & T5.4 & Test authentication flow (manual + Postman) & 2h \\ \hline
\multirow{3}{*}{US006} & \multirow{3}{*}{RBAC (Role-Based \newline Access Control)} & T6.1 & Define roles: admin, developer, viewer & 1h \\
\cline{3-5}
& & T6.2 & Add RBAC checks to middleware & 3h \\
\cline{3-5}
& & T6.3 & Test access restrictions based on role & 1h \\ \hline
\multirow{3}{*}{US007} & \multirow{3}{*}{Implement Authentification \newline tools} & T7.1 & Implement password hashing (e.g., bcrypt) & 1h \\
\cline{3-5}
& & T7.2 & Add email format validator and strong password policy & 1h \\
\cline{3-5}
& & T7.3 & Write unit tests for auth logic & 3h \\ \hline
\end{tabular}
\end{table}

\textbf{Outcomes:} Successfully established the core technology foundation with 66 hours of development work completed across 7 main user stories covering backend setup, authentication, and database implementation.

\subsubsection{Sprint 1 Diagrams}

\textbf{a) Use Case Diagram:}
\begin{figure}[H]
\centering
\includegraphics[width=0.7\textwidth,height=0.4\textheight,keepaspectratio]{rapport/media/sprint1_usecase.png}
\caption{Sprint 1a - Use Case Diagram}
\label{fig:sprint1a_usecase}
\end{figure}
\textbf{Summary:} This use case diagram illustrates the core interactions between users (admin, developer) and the system during the initial setup phase, highlighting authentication, database configuration, and API design processes.

\textbf{b) Sequence Diagram 1:}
\begin{figure}[H]
\centering
\includegraphics[width=0.7\textwidth,height=0.35\textheight,keepaspectratio]{rapport/media/sprint1_sequence1.png}
\caption{Sprint 1b - Sequence Diagram: Authentication Flow}
\label{fig:sprint1b_sequence1}
\end{figure}
\textbf{Summary:} This sequence diagram demonstrates the JWT-based authentication workflow, showing the interaction between the frontend, backend middleware, and database for secure user login and role-based access control.

\textbf{c) Sequence Diagram 2:}
\begin{figure}[H]
\centering
\includegraphics[width=0.7\textwidth,height=0.35\textheight,keepaspectratio]{rapport/media/sprint1_sequence2.png}
\caption{Sprint 1c - Sequence Diagram: Database Connection}
\label{fig:sprint1c_sequence2}
\end{figure}
\textbf{Summary:} This sequence diagram shows the PostgreSQL connection establishment process with WAL configuration, including retry mechanisms and health checks for robust database connectivity.

\textbf{d) Activity Diagram:}
\begin{figure}[H]
\centering
\includegraphics[width=0.6\textwidth,height=0.6\textheight,keepaspectratio]{rapport/media/sprint1_activity.png}
\caption{Sprint 1d - Activity Diagram: Project Setup Process}
\label{fig:sprint1d_activity}
\end{figure}
\textbf{Summary:} This activity diagram outlines the complete project initialization workflow, from Go module setup through gRPC server configuration, database schema creation, and authentication system implementation.

\subsection{Sprint 2: Real-Time Error Capture}

\textbf{Duration:} April 8, 2025 - April 21, 2025 (2 weeks)

\textbf{Sprint Goal:} Implement error ingestion and dashboard MVP for real-time monitoring.

\begin{table}[H]
\centering
\caption{Sprint 2 - Dashboard and Error Management}
\scriptsize
\begin{tabular}{|p{1cm}|>{\raggedright\arraybackslash}p{3.5cm}|p{1cm}|>{\raggedright\arraybackslash}p{7cm}|p{1cm}|}
\hline
\textbf{Story} & \textbf{Description} & \textbf{Task} & \textbf{Task Description} & \textbf{Hours} \\ \hline
\multirow{4}{*}{US008} & \multirow{4}{*}{Implement dashboard \newline metrics UI} & T8.1 & Design wireframe for metrics layout & 1h \\
\cline{3-5}
& & T8.2 & Create Vue.js components for KPI cards & 2h \\
\cline{3-5}
& & T8.3 & Integrate static data for testing UI & 2h \\
\cline{3-5}
& & T8.4 & Setup responsive layout with CSS & 1h \\ \hline
\multirow{4}{*}{US009} & \multirow{4}{*}{Develop necessary \newline APIs} & T9.1 & Define API endpoints for dashboard metrics & 3h \\
\cline{3-5}
& & T9.2 & Implement GET endpoints (/metrics, /summary) & 1h \\
\cline{3-5}
& & T9.3 & Connect to database to fetch live data & 1h \\
\cline{3-5}
& & T9.4 & Add error handling and logging & 2h \\ \hline
\multirow{4}{*}{US010} & \multirow{4}{*}{Implement Errors/Logs \newline UI} & T10.1 & Design UI layout for error/logs panel & 1h \\
\cline{3-5}
& & T10.2 & Build Vue components for logs table & 2h \\
\cline{3-5}
& & T10.3 & Add pagination and filters & 1h \\
\cline{3-5}
& & T10.4 & Connect frontend to API & 1h \\ \hline
\end{tabular}
\end{table}

\textbf{Outcomes:} Delivered a functional dashboard capable of real-time error monitoring with 22 hours of development work across 3 user stories.

\subsubsection{Sprint 2 Diagrams}

\textbf{a) Use Case Diagram:}
\begin{figure}[H]
\centering
\includegraphics[width=0.7\textwidth,height=0.4\textheight,keepaspectratio]{rapport/media/sprint2_usecase.png}
\caption{Sprint 2a - Use Case Diagram}
\label{fig:sprint2a_usecase}
\end{figure}
\textbf{Summary:} This use case diagram depicts the real-time error monitoring capabilities, showing how developers and administrators interact with the dashboard to view metrics, analyze error logs, and monitor system performance.

\textbf{b) Sequence Diagram 1:}
\begin{figure}[H]
\centering
\includegraphics[width=0.7\textwidth,height=0.35\textheight,keepaspectratio]{rapport/media/sprint2_sequence1.png}
\caption{Sprint 2b - Sequence Diagram: Dashboard Data Flow}
\label{fig:sprint2b_sequence1}
\end{figure}
\textbf{Summary:} This sequence diagram illustrates the data flow from the backend APIs to the Vue.js dashboard components, showing how real-time metrics and KPIs are fetched and displayed to users.

\textbf{c) Sequence Diagram 2:}
\begin{figure}[H]
\centering
\includegraphics[width=0.7\textwidth,height=0.35\textheight,keepaspectratio]{rapport/media/sprint2_sequence2.png}
\caption{Sprint 2c - Sequence Diagram: Error Log Retrieval}
\label{fig:sprint2c_sequence2}
\end{figure}
\textbf{Summary:} This sequence diagram demonstrates the error log retrieval process, including pagination, filtering, and real-time updates for the error monitoring interface.

\textbf{d) Activity Diagram:}
\begin{figure}[H]
\centering
\includegraphics[width=0.6\textwidth,height=0.6\textheight,keepaspectratio]{rapport/media/sprint2_activity.png}
\caption{Sprint 2d - Activity Diagram: Real-Time Error Monitoring}
\label{fig:sprint2d_activity}
\end{figure}
\textbf{Summary:} This activity diagram shows the complete workflow for implementing real-time error monitoring, from dashboard UI design through API development and data integration.

\subsection{Sprint 3: DevOps Foundation}

\textbf{Duration:} April 22, 2025 - May 5, 2025 (2 weeks)

\textbf{Sprint Goal:} Establish CI/CD pipeline and DevOps automation infrastructure.

\begin{table}[H]
\centering
\caption{Sprint 3 - DevOps Pipeline Implementation}
\scriptsize
\begin{tabular}{|p{1cm}|>{\raggedright\arraybackslash}p{3.5cm}|p{1cm}|>{\raggedright\arraybackslash}p{7cm}|p{1cm}|}
\hline
\textbf{Story} & \textbf{Description} & \textbf{Task} & \textbf{Task Description} & \textbf{Hours} \\ \hline
\multirow{4}{*}{US013} & \multirow{4}{*}{Implement pipeline \newline dashboard} & T13.1 & Define UI/UX requirements for pipeline dashboard & 4h \\
\cline{3-5}
& & T13.2 & Set up frontend components for pipeline visualization & 2h \\
\cline{3-5}
& & T13.3 & Implement backend endpoints for pipeline data & 4h \\
\cline{3-5}
& & T13.4 & Integrate real-time updates (WebSockets) & 3h \\ \hline
\multirow{4}{*}{US015} & \multirow{4}{*}{Implement pipeline \newline tools} & T15.1 & Evaluate and choose CI/CD tools & 4h \\
\cline{3-5}
& & T15.2 & Configure GitHub Actions with repository & 4h \\
\cline{3-5}
& & T15.3 & Define pipeline stages (build, test, deploy) & 3h \\
\cline{3-5}
& & T15.4 & Integrate automated testing & 3h \\ \hline
\end{tabular}
\end{table}

\textbf{Outcomes:} Achieved full DevOps automation with 27 hours of development work across 2 main user stories.

\subsubsection{Sprint 3 Diagrams}

\textbf{a) Use Case Diagram:}
\begin{figure}[H]
\centering
\includegraphics[width=0.7\textwidth,height=0.4\textheight,keepaspectratio]{rapport/media/sprint3_usecase.png}
\caption{Sprint 3a - Use Case Diagram}
\label{fig:sprint3a_usecase}
\end{figure}
\textbf{Summary:} This use case diagram shows the DevOps automation interactions, illustrating how developers and administrators manage CI/CD pipelines, monitor deployments, and configure automated testing workflows.

\textbf{b) Sequence Diagram 1:}
\begin{figure}[H]
\centering
\includegraphics[width=0.7\textwidth,height=0.35\textheight,keepaspectratio]{rapport/media/sprint3_sequence1.png}
\caption{Sprint 3b - Sequence Diagram: CI/CD Pipeline Execution}
\label{fig:sprint3b_sequence1}
\end{figure}
\textbf{Summary:} This sequence diagram demonstrates the CI/CD pipeline execution flow using GitHub Actions, showing the interaction between repository commits, build processes, testing phases, and deployment stages.

\textbf{c) Sequence Diagram 2:}
\begin{figure}[H]
\centering
\includegraphics[width=0.7\textwidth,height=0.35\textheight,keepaspectratio]{rapport/media/sprint3_sequence2.png}
\caption{Sprint 3c - Sequence Diagram: Pipeline Monitoring}
\label{fig:sprint3c_sequence2}
\end{figure}
\textbf{Summary:} This sequence diagram illustrates the real-time pipeline monitoring system, showing how WebSockets enable live updates of build status, test results, and deployment progress.

\textbf{d) Activity Diagram:}
\begin{figure}[H]
\centering
\includegraphics[width=0.6\textwidth,height=0.6\textheight,keepaspectratio]{rapport/media/sprint3_activity.png}
\caption{Sprint 3d - Activity Diagram: DevOps Pipeline Setup}
\label{fig:sprint3d_activity}
\end{figure}
\textbf{Summary:} This activity diagram outlines the complete DevOps pipeline setup process, from tool evaluation and GitHub Actions configuration to automated testing integration and monitoring dashboard implementation.

\subsection{Sprint 4: Error Classification \& AI Fixes}

\textbf{Duration:} May 6, 2025 - May 19, 2025 (2 weeks)

\textbf{Sprint Goal:} Integrate AI-powered error analysis and automated correction capabilities.

\begin{table}[H]
\centering
\caption{Sprint 4 - AI Integration and Error Correction}
\scriptsize
\begin{tabular}{|p{1cm}|>{\raggedright\arraybackslash}p{3.5cm}|p{1cm}|>{\raggedright\arraybackslash}p{7cm}|p{1cm}|}
\hline
\textbf{Story} & \textbf{Description} & \textbf{Task} & \textbf{Task Description} & \textbf{Hours} \\ \hline
\multirow{2}{*}{US017} & \multirow{2}{*}{Implement AI \newline model} & T17.1 & Integrate DeepSeek API & 2h \\
\cline{3-5}
& & T17.2 & Integrate model inference into backend & 4h \\ \hline
\multirow{2}{*}{US018} & \multirow{2}{*}{Tag errors with \newline suggested fixes} & T18.1 & Implement tagging system for errors & 4h \\
\cline{3-5}
& & T18.2 & Provide structured metadata for developers & 4h \\ \hline
\multirow{2}{*}{US019} & \multirow{2}{*}{Implement automated \newline code fixes} & T19.1 & Build mechanism to apply code fixes & 4h \\
\cline{3-5}
& & T19.2 & Ensure rollback strategy for incorrect fixes & 2h \\ \hline
\end{tabular}
\end{table}

\textbf{Outcomes:} Successfully implemented AI-driven error correction with 22 hours of development work across 3 user stories.

\subsubsection{Sprint 4 Diagrams}

\textbf{a) Use Case Diagram:}
\begin{figure}[H]
\centering
\includegraphics[width=0.7\textwidth,height=0.4\textheight,keepaspectratio]{rapport/media/sprint4_usecase.png}
\caption{Sprint 4a - Use Case Diagram}
\label{fig:sprint4a_usecase}
\end{figure}
\textbf{Summary:} This use case diagram illustrates the AI-powered error analysis system, showing how developers interact with automated error classification, fix suggestions, and code correction capabilities.

\textbf{b) Sequence Diagram 1:}
\begin{figure}[H]
\centering
\includegraphics[width=0.7\textwidth,height=0.35\textheight,keepaspectratio]{rapport/media/sprint4_sequence1.png}
\caption{Sprint 4b - Sequence Diagram: AI Model Integration}
\label{fig:sprint4b_sequence1}
\end{figure}
\textbf{Summary:} This sequence diagram shows the DeepSeek API integration process, demonstrating how error data is sent to the AI model, processed for analysis, and returned with classification results and fix suggestions.

\textbf{c) Sequence Diagram 2:}
\begin{figure}[H]
\centering
\includegraphics[width=0.7\textwidth,height=0.35\textheight,keepaspectratio]{rapport/media/sprint4_sequence2.png}
\caption{Sprint 4c - Sequence Diagram: Automated Code Fixes}
\label{fig:sprint4c_sequence2}
\end{figure}
\textbf{Summary:} This sequence diagram illustrates the automated code fix application process, including the rollback mechanism for incorrect fixes and the validation workflow for successful corrections.

\textbf{d) Activity Diagram:}
\begin{figure}[H]
\centering
\includegraphics[width=0.6\textwidth,height=0.6\textheight,keepaspectratio]{rapport/media/sprint4_activity.png}
\caption{Sprint 4d - Activity Diagram: Error Classification \& AI Fixes}
\label{fig:sprint4d_activity}
\end{figure}
\textbf{Summary:} This activity diagram details the complete AI-driven error correction workflow, from error detection and classification through automated fix generation and rollback strategy implementation.

\subsection{Sprint 5: Alerting \& Notifications}

\textbf{Duration:} May 20, 2025 - June 2, 2025 (2 weeks)

\textbf{Sprint Goal:} Implement comprehensive notification and alerting system.

\begin{table}[H]
\centering
\caption{Sprint 5 - Notifications and Alert Management}
\scriptsize
\begin{tabular}{|p{1cm}|>{\raggedright\arraybackslash}p{3.5cm}|p{1cm}|>{\raggedright\arraybackslash}p{7cm}|p{1cm}|}
\hline
\textbf{Story} & \textbf{Description} & \textbf{Task} & \textbf{Task Description} & \textbf{Hours} \\ \hline
\multirow{6}{*}{US022} & \multirow{6}{*}{Configure Tools \newline integrations} & T22.1 & Design integration architecture with external tools & 2h \\
\cline{3-5}
& & T22.2 & Implement Slack webhook API & 1h \\
\cline{3-5}
& & T22.3 & Implement Teams webhook API & 1h \\
\cline{3-5}
& & T22.4 & Implement Discord webhook integration & 2h \\
\cline{3-5}
& & T22.5 & Create generic webhook interface & 2h \\
\cline{3-5}
& & T22.6 & Test webhook delivery reliability & 2h \\ \hline
\multirow{6}{*}{US023} & \multirow{6}{*}{Integrations notifications \newline System} & T23.1 & Design notification dispatcher architecture & 1h \\
\cline{3-5}
& & T23.2 & Create notification templates engine & 2h \\
\cline{3-5}
& & T23.3 & Implement routing logic per integration type & 2h \\
\cline{3-5}
& & T23.4 & Build retry mechanism for failed notifications & 1h \\
\cline{3-5}
& & T23.5 & Unit test notification service & 1h \\
\cline{3-5}
& & T23.6 & Verify end-to-end delivery to integrated tools & 1h \\ \hline
\multirow{5}{*}{US024} & \multirow{5}{*}{Implement Billing \newline Alerts} & T24.1 & Identify billing thresholds (quota, over-usage, abnormal costs) & 2h \\
\cline{3-5}
& & T24.2 & Implement rule engine for billing alerts & 1h \\
\cline{3-5}
& & T24.3 & Connect billing system data to alert service & 2h \\
\cline{3-5}
& & T24.4 & Create alert templates (email/SMS/integration) & 2h \\
\cline{3-5}
& & T24.5 & Test alert triggering with simulated billing events & 2h \\ \hline
\multirow{5}{*}{US025} & \multirow{5}{*}{Throttle \newline Alerts} & T25.1 & Analyze alert flood scenarios & 2h \\
\cline{3-5}
& & T25.2 & Implement throttling middleware in alert pipeline & 3h \\
\cline{3-5}
& & T25.3 & Store last-sent timestamp per error type (Redis/DB cache) & 2h \\
\cline{3-5}
& & T25.4 & Enforce max frequency (1/min per type) & 1h \\
\cline{3-5}
& & T25.5 & Add logging for suppressed alerts & 2h \\ \hline
\multirow{6}{*}{US026} & \multirow{6}{*}{Handle Alerting \newline Rules} & T26.1 & Design rules schema (conditions, thresholds, channels) & 2h \\
\cline{3-5}
& & T26.2 & Build CRUD API for alert rules (create/update/delete) & 2h \\
\cline{3-5}
& & T26.3 & Implement rules evaluation engine & 1h \\
\cline{3-5}
& & T26.4 & Store rules in DB & 3h \\
\cline{3-5}
& & T26.5 & Integrate rules engine with notification dispatcher & 2h \\
\cline{3-5}
& & T26.6 & Build UI (basic or API endpoints) to manage rules & 4h \\ \hline
\end{tabular}
\end{table}

\textbf{Outcomes:} Successfully implemented comprehensive notification system with 48 hours of development work across 5 user stories.

\subsubsection{Sprint 5 Diagrams}

\textbf{a) Use Case Diagram:}
\begin{figure}[H]
\centering
\includegraphics[width=0.7\textwidth,height=0.4\textheight,keepaspectratio]{rapport/media/sprint5_usecase.png}
\caption{Sprint 5a - Use Case Diagram}
\label{fig:sprint5a_usecase}
\end{figure}
\textbf{Summary:} This use case diagram demonstrates the comprehensive notification and alerting system, showing how administrators configure alert rules, manage integrations, and users receive notifications through multiple channels.

\textbf{b) Sequence Diagram 1:}
\begin{figure}[H]
\centering
\includegraphics[width=0.7\textwidth,height=0.35\textheight,keepaspectratio]{rapport/media/sprint5_sequence1.png}
\caption{Sprint 5b - Sequence Diagram: Notification System Flow}
\label{fig:sprint5b_sequence1}
\end{figure}
\textbf{Summary:} This sequence diagram illustrates the multi-channel notification flow, showing how alerts are processed through the dispatcher, routed to appropriate channels (Slack, Teams, Discord), and delivered with retry mechanisms.

\textbf{c) Sequence Diagram 2:}
\begin{figure}[H]
\centering
\includegraphics[width=0.7\textwidth,height=0.35\textheight,keepaspectratio]{rapport/media/sprint5_sequence2.png}
\caption{Sprint 5c - Sequence Diagram: Alert Rules Management}
\label{fig:sprint5c_sequence2}
\end{figure}
\textbf{Summary:} This sequence diagram shows the alert rules management process, including CRUD operations for rules configuration, threshold evaluation, and billing alert implementation with throttling mechanisms.

\textbf{d) Activity Diagram:}
\begin{figure}[H]
\centering
\includegraphics[width=0.6\textwidth,height=0.6\textheight,keepaspectratio]{rapport/media/sprint5_activity.png}
\caption{Sprint 5d - Activity Diagram: Alerting \& Notifications}
\label{fig:sprint5d_activity}
\end{figure}
\textbf{Summary:} This activity diagram outlines the complete alerting and notification implementation workflow, from tools integration setup through billing alerts and throttling mechanisms to comprehensive rule management.

\subsection{Sprint 6: Data Protection \& Payments}

\textbf{Duration:} June 3, 2025 - June 16, 2025 (2 weeks)

\textbf{Sprint Goal:} Implement security, compliance, and billing functionality.

\begin{table}[H]
\centering
\caption{Sprint 6 - Security and Payment Integration}
\scriptsize
\begin{tabular}{|p{1cm}|>{\raggedright\arraybackslash}p{3.5cm}|p{1cm}|>{\raggedright\arraybackslash}p{7cm}|p{1cm}|}
\hline
\textbf{Story} & \textbf{Description} & \textbf{Task} & \textbf{Task Description} & \textbf{Hours} \\ \hline
\multirow{5}{*}{US027} & \multirow{5}{*}{Encrypt sensitive data \newline using (AES-256)} & T027.1 & Analyze and identify sensitive data fields requiring encryption & 8h \\
\cline{3-5}
& & T027.2 & Implement AES-256 encryption for database fields & 12h \\
\cline{3-5}
& & T027.3 & Develop key management system for encryption keys & 10h \\
\cline{3-5}
& & T027.4 & Create data encryption/decryption utilities & 8h \\
\cline{3-5}
& & T027.5 & Test encryption implementation and performance & 7h \\ \hline
\multirow{5}{*}{US028} & \multirow{5}{*}{Implement GDPR-compliant \newline audit logging} & T028.1 & Define GDPR audit logging requirements and data scope & 4h \\
\cline{3-5}
& & T028.2 & Design audit log schema and storage structure & 5h \\
\cline{3-5}
& & T028.3 & Implement audit logging middleware/interceptors & 8h \\
\cline{3-5}
& & T028.4 & Create log retrieval and export functionality & 6h \\
\cline{3-5}
& & T028.5 & Implement log retention and deletion policies & 5h \\ \hline
\multirow{5}{*}{US029} & \multirow{5}{*}{Compute the usage \newline of system} & T029.1 & Define usage metrics and tracking requirements & 10h \\
\cline{3-5}
& & T029.2 & Implement usage data collection mechanisms & 12h \\
\cline{3-5}
& & T029.3 & Create usage analytics and reporting module & 15h \\
\cline{3-5}
& & T029.4 & Develop usage dashboard and visualization & 8h \\
\cline{3-5}
& & T029.5 & Implement usage alerts and notifications & 5h \\ \hline
\multirow{6}{*}{US030} & \multirow{6}{*}{Handle payments \newline services} & T030.1 & Research and select payment gateway integration & 6h \\
\cline{3-5}
& & T030.2 & Implement payment processing workflow & 10h \\
\cline{3-5}
& & T030.3 & Create payment transaction logging and tracking & 8h \\
\cline{3-5}
& & T030.4 & Develop refund and cancellation handling & 6h \\
\cline{3-5}
& & T030.5 & Implement payment security and PCI compliance measures & 12h \\
\cline{3-5}
& & T030.6 & Test payment integration end-to-end & 8h \\ \hline
\multirow{5}{*}{US031} & \multirow{5}{*}{Implement role-based \newline permissions} & T031.1 & Define role hierarchy and permission matrix & 8h \\
\cline{3-5}
& & T031.2 & Create database schema for roles and permissions & 6h \\
\cline{3-5}
& & T031.3 & Implement permission checking middleware & 10h \\
\cline{3-5}
& & T031.4 & Develop user-role assignment interface & 8h \\
\cline{3-5}
& & T031.5 & Create permission testing and validation suite & 6h \\ \hline
\end{tabular}
\end{table}

\textbf{Outcomes:} Achieved enterprise-grade security and compliance with 183 hours of development work across 5 user stories.

\subsubsection{Sprint 6 Diagrams}

\textbf{a) Use Case Diagram:}
\begin{figure}[H]
\centering
\includegraphics[width=0.7\textwidth,height=0.4\textheight,keepaspectratio]{rapport/media/sprint6_usecase.png}
\caption{Sprint 6a - Use Case Diagram}
\label{fig:sprint6a_usecase}
\end{figure}
\textbf{Summary:} This use case diagram illustrates the enterprise security and compliance features, showing how administrators manage data encryption, GDPR compliance, payment processing, and role-based permissions within the system.

\textbf{b) Sequence Diagram 1:}
\begin{figure}[H]
\centering
\includegraphics[width=0.7\textwidth,height=0.35\textheight,keepaspectratio]{rapport/media/sprint6_sequence1.png}
\caption{Sprint 6b - Sequence Diagram: Data Encryption Process}
\label{fig:sprint6b_sequence1}
\end{figure}
\textbf{Summary:} This sequence diagram demonstrates the AES-256 encryption implementation, showing how sensitive data is encrypted/decrypted, key management processes, and secure storage mechanisms for data protection.

\textbf{c) Sequence Diagram 2:}
\begin{figure}[H]
\centering
\includegraphics[width=0.7\textwidth,height=0.35\textheight,keepaspectratio]{rapport/media/sprint6_sequence2.png}
\caption{Sprint 6c - Sequence Diagram: Payment Processing}
\label{fig:sprint6c_sequence2}
\end{figure}
\textbf{Summary:} This sequence diagram illustrates the secure payment processing workflow, including gateway integration, transaction logging, PCI compliance measures, and refund/cancellation handling mechanisms.

\textbf{d) Activity Diagram:}
\begin{figure}[H]
\centering
\includegraphics[width=0.6\textwidth,height=0.6\textheight,keepaspectratio]{rapport/media/sprint6_activity.png}
\caption{Sprint 6d - Activity Diagram: Data Protection \& Payments}
\label{fig:sprint6d_activity}
\end{figure}
\textbf{Summary:} This activity diagram details the comprehensive security implementation process, covering data encryption, GDPR audit logging, usage tracking, payment integration, and role-based access control setup.

\subsection{Sprint 7: SDKs \& Plugins}

\textbf{Duration:} June 17, 2025 - June 30, 2025 (2 weeks)

\textbf{Sprint Goal:} Develop client SDKs and comprehensive documentation.

\begin{table}[H]
\centering
\caption{Sprint 7 - SDK Development and Documentation}
\scriptsize
\begin{tabular}{|p{1cm}|>{\raggedright\arraybackslash}p{3.5cm}|p{1cm}|>{\raggedright\arraybackslash}p{7cm}|p{1cm}|}
\hline
\textbf{Story} & \textbf{Description} & \textbf{Task} & \textbf{Task Description} & \textbf{Hours} \\ \hline
\multirow{5}{*}{US032} & \multirow{5}{*}{Create Plugin \newline for NodeJs} & T032.1 & Design plugin architecture and API interface & 6h \\
\cline{3-5}
& & T032.2 & Implement core plugin functionality and methods & 10h \\
\cline{3-5}
& & T032.3 & Write unit tests and integration tests for the plugin & 8h \\
\cline{3-5}
& & T032.4 & Package and publish plugin to NPM registry & 4h \\
\cline{3-5}
& & T032.5 & Create basic usage examples and README & 4h \\ \hline
\multirow{5}{*}{US033} & \multirow{5}{*}{Create SDK for \newline Flutter/Dart} & T033.1 & Design SDK architecture and data models (Dart Classes) & 8h \\
\cline{3-5}
& & T033.2 & Implement API client and network communication layer & 10h \\
\cline{3-5}
& & T033.3 & Develop core SDK features and methods & 10h \\
\cline{3-5}
& & T033.4 & Write comprehensive unit and widget tests & 8h \\
\cline{3-5}
& & T033.5 & Document the SDK API and publish to pub.dev & 6h \\ \hline
\multirow{5}{*}{US034} & \multirow{5}{*}{Handle Service \newline Activation} & T034.1 & Design service activation workflow (trial, paid, etc.) & 6h \\
\cline{3-5}
& & T034.2 & Implement activation endpoint and status tracking & 8h \\
\cline{3-5}
& & T034.3 & Develop license key generation and validation logic & 8h \\
\cline{3-5}
& & T034.4 & Create admin interface for managing activations & 6h \\
\cline{3-5}
& & T034.5 & Test activation/deactivation scenarios end-to-end & 6h \\ \hline
\multirow{5}{*}{US035} & \multirow{5}{*}{Create Documentation for \newline exploring Service} & T035.1 & Outline documentation structure and user journeys & 5h \\
\cline{3-5}
& & T035.2 & Write "Getting Started" guide and installation instructions & 6h \\
\cline{3-5}
& & T035.3 & Create comprehensive API reference documentation & 12h \\
\cline{3-5}
& & T035.4 & Develop tutorials and code samples for common use cases & 10h \\
\cline{3-5}
& & T035.5 & Set up and deploy documentation site (e.g., GitBook, Docusaurus) & 5h \\ \hline
\end{tabular}
\end{table}

\textbf{Outcomes:} Delivered complete SDK ecosystem and documentation with 142 hours of development work across 4 user stories.

\subsubsection{Sprint 7 Diagrams}

\textbf{a) Use Case Diagram:}
\begin{figure}[H]
\centering
\includegraphics[width=0.7\textwidth,height=0.4\textheight,keepaspectratio]{rapport/media/sprint7_usecase.png}
\caption{Sprint 7a - Use Case Diagram}
\label{fig:sprint7a_usecase}
\end{figure}
\textbf{Summary:} This use case diagram shows the SDK and plugin ecosystem, illustrating how developers integrate Node.js plugins, Flutter/Dart SDKs, manage service activation, and access comprehensive documentation for system integration.

\textbf{b) Sequence Diagram 1:}
\begin{figure}[H]
\centering
\includegraphics[width=0.7\textwidth,height=0.35\textheight,keepaspectratio]{rapport/media/sprint7_sequence1.png}
\caption{Sprint 7b - Sequence Diagram: SDK Integration}
\label{fig:sprint7b_sequence1}
\end{figure}
\textbf{Summary:} This sequence diagram demonstrates the SDK integration process, showing how external applications use the Node.js plugin and Flutter/Dart SDK to communicate with the ErrorZen API endpoints.

\textbf{c) Sequence Diagram 2:}
\begin{figure}[H]
\centering
\includegraphics[width=0.7\textwidth,height=0.35\textheight,keepaspectratio]{rapport/media/sprint7_sequence2.png}
\caption{Sprint 7c - Sequence Diagram: Service Activation}
\label{fig:sprint7c_sequence2}
\end{figure}
\textbf{Summary:} This sequence diagram illustrates the service activation workflow, including license key generation, validation logic, trial/paid service management, and admin interface interactions.

\textbf{d) Activity Diagram:}
\begin{figure}[H]
\centering
\includegraphics[width=0.6\textwidth,height=0.6\textheight,keepaspectratio]{rapport/media/sprint7_activity.png}
\caption{Sprint 7d - Activity Diagram: SDK Development \& Documentation}
\label{fig:sprint7d_activity}
\end{figure}
\textbf{Summary:} This activity diagram outlines the complete SDK development and documentation process, from plugin architecture design through testing, publishing, and comprehensive documentation site deployment.

\subsection{Sprint Summary and Metrics}

\begin{table}[H]
\centering
\caption{Sprint Summary and Effort Distribution}
\footnotesize
\begin{tabular}{|p{1.5cm}|p{1.5cm}|p{1.5cm}|>{\raggedright\arraybackslash}p{3cm}|>{\raggedright\arraybackslash}p{4cm}|}
\hline
\textbf{Sprint} & \textbf{Stories} & \textbf{Hours} & \textbf{Focus Area} & \textbf{Key Achievement} \\ \hline
Sprint 1 & 7 & 66 & Infrastructure & Backend and auth foundation \\ \hline
Sprint 2 & 3 & 22 & Frontend & Real-time dashboard \\ \hline
Sprint 3 & 2 & 27 & DevOps & CI/CD automation \\ \hline
Sprint 4 & 3 & 22 & AI/ML & Error auto-correction \\ \hline
Sprint 5 & 5 & 48 & Notifications & Multi-channel alerts \\ \hline
Sprint 6 & 5 & 183 & Security & Enterprise compliance \\ \hline
Sprint 7 & 4 & 142 & Integration & SDK ecosystem \\ \hline
\textbf{Total} & \textbf{29} & \textbf{510} & \textbf{Complete} & \textbf{Production-ready MVP} \\ \hline
\end{tabular}
\end{table}

\subsection{Lessons Learned}

\subsubsection{Technical Insights}
\begin{itemize}
\item Go's concurrency model proved excellent for real-time error processing
\item PostgreSQL's WAL feature was crucial for reliable error logging
\item Vue.js provided the right balance of simplicity and functionality for the dashboard
\item DeepSeek API integration exceeded expectations for AI-powered error correction
\end{itemize}

\subsubsection{Process Improvements}
\begin{itemize}
\item Two-week sprints provided optimal balance between planning and flexibility
\item RAD methodology acceleration reduced time-to-market by approximately 30\%
\item Continuous integration prevented integration issues and maintained code quality
\item Regular retrospectives led to meaningful process improvements
\end{itemize}



\subsubsection{Challenges Overcome}
\begin{itemize}
\item Initial complexity of gRPC configuration - resolved through better documentation
\item AI model integration latency - optimized through caching and async processing
\item Multi-platform SDK compatibility - addressed through comprehensive testing
\item DevOps pipeline stability - improved through better error handling and monitoring
\end{itemize}

\subsection{Future Enhancements}

Based on the sprint outcomes and user feedback, the following enhancements are planned for future iterations:

\begin{itemize}
\item Advanced machine learning models for better error prediction
\item Extended language support for additional programming frameworks
\item Enhanced mobile application monitoring capabilities
\item Integration with more third-party development tools
\item Advanced analytics and reporting features
\end{itemize}