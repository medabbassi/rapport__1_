% Section 2: Project Management & Methodology

\section{Project Management \& Methodology}

\subsection{Overview of Agile and Scrum}

Agile and Scrum emphasize short cycles, collaboration, and rapid feedback. I used Scrum to manage changing requirements and keep development structured but flexible.

\subsection{Adapting Scrum Roles in a RAD Context}

In a traditional Scrum team, roles are clearly defined:

\begin{itemize}
\item \textbf{Product Owner:} Represents the client, prioritises the product backlog, and validates features
\item \textbf{Scrum Master:} Ensures adherence to Agile principles, removes blockers, and facilitates ceremonies
\item \textbf{Development Team:} Delivers functional increments each sprint
\end{itemize}

\subsubsection*{My RAD (Rapid Application Development) Adaptation (Solo/Small Team)}

In a solo RAD context, I merged Scrum roles for speed and direct feedback. I acted as Product Owner, Scrum Master, and Developer, quickly adjusting priorities and validating requirements using tools like Miro and Figma. Technical spikes were strictly timeboxed, and I automated testing and CI/CD for fast delivery. This approach reduced delays and improved ownership.

\begin{figure}[H]
\centering
\includegraphics[width=0.7\textwidth,keepaspectratio]{rapport/media/communication_diag.png}
\caption{RAD/Scrum Hybrid Workflow}
\label{fig:rad_scrum}
\end{figure}

\subsection{Scrum Ceremonies}

I adapted Scrum ceremonies for solo work: Sprint Planning set clear goals, daily stand-ups tracked progress, and reviews/retrospectives ensured continuous improvement.

\subsection{Tools Used}

I used Trello for sprint boards, GitHub for code, Notion for docs, Figma for UI, and Draw.io for diagrams. VS Code, GoLand, and Docker covered development and deployment needs.

\begin{figure}[H]
\centering
\includegraphics[width=0.7\textwidth,keepaspectratio]{rapport/media/image2.png}
\caption{Development Tools and Technologies}
\label{fig:dev_tools}
\end{figure}

\subsection{Sprint Length and Structure}

Each sprint lasted two weeks: planning, daily development, review, and retrospective.

\subsection{Project Timeline}

The project ran for 7 sprints, summarized below:

\begin{table}[H]
\centering
\caption{Sprint Timeline and Goals}
\label{tab:sprints}
\small
\begin{tabular}{|p{2cm}|p{3cm}|p{8cm}|}
\hline
\textbf{Sprint} & \textbf{Duration} & \textbf{Goal} \\ \hline
Sprint 1 & Week 1--2 & Core Infrastructure Setup: Backend (Go/gRPC), PostgreSQL, CI/CD pipeline \\ \hline
Sprint 2 & Week 3--4 & Error Ingestion \& Dashboard MVP: Real-time error capture + Vue.js UI \\ \hline
Sprint 3 & Week 5--6 & AI Integration: PyTorch model training for error classification \\ \hline
Sprint 4 & Week 7--8 & Auto-Correction Engine: AI-driven fixes + unit test generation \\ \hline
Sprint 5 & Week 9--10 & DevOps Automation: GitHub Actions workflows, K8S deployment \\ \hline
Sprint 6 & Week 11--12 & SDK \& Integrations: Sentry/Firebase compatibility, Slack alerts \\ \hline
Sprint 7 & Week 13--14 & Polish \& Scalability: Load testing, security audits, documentation \\ \hline
\end{tabular}
\end{table}

\subsection{Gantt Chart}

\begin{figure}[H]
\centering
\includegraphics[width=0.9\textwidth,keepaspectratio]{rapport/media/image4.png}
\caption{Gantt Chart - Project Timeline}
\label{fig:gantt}
\end{figure}