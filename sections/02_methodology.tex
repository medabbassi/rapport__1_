% Section 2: Project Management & Methodology

\section{Project Management \& Methodology}

\subsection{Overview of Agile and Scrum}

Agile is a flexible software development methodology that emphasises iterative progress, collaboration, and user feedback. Scrum is a widely used Agile framework characterised by short, time-boxed development cycles called sprints, daily team meetings, and continuous delivery of value.

In this project, Scrum was adopted to manage changing requirements and ensure a structured yet adaptable development process.

\subsection{Adapting Scrum Roles in a RAD Context}

In a traditional Scrum team, roles are clearly defined:

\begin{itemize}
\item \textbf{Product Owner:} Represents the client, prioritises the product backlog, and validates features
\item \textbf{Scrum Master:} Ensures adherence to Agile principles, removes blockers, and facilitates ceremonies
\item \textbf{Development Team:} Delivers functional increments each sprint
\end{itemize}

\subsubsection*{My RAD (Rapid Application Development) Adaptation (Solo/Small Team)}

Working in a fast-paced, iterative RAD environment (where speed and client feedback are critical), I merged these roles to maximise efficiency:

\textbf{Product Owner + RAD Analyst}
\begin{itemize}
\item Direct client collaboration to capture just-in-time requirements and dynamically adjust the backlog
\item MVP-driven prioritisation (typical of RAD cycles), with client feedback integrated after each prototype
\item Use of visual tools (e.g., Miro, clickable mockups) for rapid requirement validation
\end{itemize}

\textbf{Scrum Master + Technical Coordinator}
\begin{itemize}
\item Self-facilitated ceremonies (e.g., solo planning poker via relative effort points, retrospectives focused on continuous improvement)
\item Proactive risk/blocker management with strict timeboxing (e.g., capping technical spikes at 2 hours)
\item Leveraged RAD tools (low-code platforms, code generators) to accelerate development cycles
\end{itemize}

\textbf{Full-Stack Developer + Integrator}
\begin{itemize}
\item Feature-driven implementation with technical spikes for Proof of Concepts (POCs)
\item Automated testing and CI/CD pipelines for real-time validation of each increment
\item Incremental documentation via living docs (e.g., updated README.md per commit)
\end{itemize}

\textbf{Benefits of This Hybrid Approach:}
\begin{itemize}
\item Faster time-to-market by eliminating role synchronisation delays
\item Greater flexibility to pivot based on client feedback (core RAD principle)
\item Stronger ownership across the entire development lifecycle
\end{itemize}

\subsection{Scrum Ceremonies}

The following ceremonies were adopted:

\begin{itemize}
\item \textbf{Sprint Planning:} Defined sprint goals and selected backlog items
\item \textbf{Daily Stand-ups:} Brief updates on progress and blockers (documented if solo)
\item \textbf{Sprint Review:} Demonstrated completed features to stakeholders or self-evaluated
\item \textbf{Sprint Retrospective:} Identified improvements to apply in the next sprint
\end{itemize}

\subsection{Tools Used}

\begin{itemize}
\item \textbf{Project Management:} Trello (backlog, sprint boards)
\item \textbf{Version Control:} Git + GitHub
\item \textbf{Documentation:} Notion / Google Docs
\item \textbf{Design \& Wireframing:} Figma / Draw.io
\item \textbf{Code Editor:} VS Code / Android Studio / GoLand / PgAdmin / Docker Desktop
\end{itemize}

\subsection{Sprint Length and Structure}

Each sprint lasted \textbf{two weeks} and followed this structure:

\begin{itemize}
\item Day 1: Sprint Planning
\item Day 2--12: Development + Daily Stand-ups
\item Day 13: Sprint Review (demo or milestone check)
\item Day 14: Sprint Retrospective
\end{itemize}

\subsection{Project Timeline}

A total of \textbf{7 sprints} were conducted, as shown in the timeline below:

\begin{table}[H]
\centering
\caption{Sprint Timeline and Goals}
\label{tab:sprints}
\small
\begin{tabular}{|p{2cm}|p{3cm}|p{8cm}|}
\hline
\textbf{Sprint} & \textbf{Duration} & \textbf{Goal} \\ \hline
Sprint 1 & Week 1--2 & Core Infrastructure Setup: Backend (Go/gRPC), PostgreSQL, CI/CD pipeline \\ \hline
Sprint 2 & Week 3--4 & Error Ingestion \& Dashboard MVP: Real-time error capture + Vue.js UI \\ \hline
Sprint 3 & Week 5--6 & AI Integration: PyTorch model training for error classification \\ \hline
Sprint 4 & Week 7--8 & Auto-Correction Engine: AI-driven fixes + unit test generation \\ \hline
Sprint 5 & Week 9--10 & DevOps Automation: GitHub Actions workflows, K8S deployment \\ \hline
Sprint 6 & Week 11--12 & SDK \& Integrations: Sentry/Firebase compatibility, Slack alerts \\ \hline
Sprint 7 & Week 13--14 & Polish \& Scalability: Load testing, security audits, documentation \\ \hline
\end{tabular}
\end{table}

\subsection{Gantt Chart}

\begin{figure}[H]
\centering
\includegraphics[width=0.9\textwidth,keepaspectratio]{rapport/media/image4.png}
\caption{Gantt Chart - Project Timeline}
\label{fig:gantt}
\end{figure}