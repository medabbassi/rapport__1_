% Section 4: Requirement Analysis

\section{Requirement Analysis}

\subsection{Introduction}

This chapter outlines the system's requirements, including both functional and non-functional aspects. It is the foundation upon which the system's design and implementation are based. The requirements were collected through meetings with stakeholders, analysis of the domain, and study of existing systems.

\subsection{Client Expectations}

These requirements describe the main functionalities that the system must offer.

\subsubsection{Error Detection and Handling}

Real-time error capture across all platforms (React, Go, Flutter) centralized in an interactive dashboard with intelligent filtering by criticality.

\subsubsection{AI Error Analysis and Correction}

Automatic AI analysis identifying root causes, proposing context-aware solutions, and generating unit tests to validate fixes.

\subsubsection{DevOps Automation}

Full automation: error detection → testing → CI/CD deployment → production tracking, all without manual intervention.

\begin{figure}[H]
\centering
\includegraphics[width=0.75\textwidth,keepaspectratio]{rapport/media/notification_config.png}
\caption{Automated Deployment Pipeline}
\label{fig:auto_deploy}
\end{figure}

\subsubsection{Integration with Other Tools}

I recognized that teams already use various tools and platforms, so ErrorZen needed to integrate seamlessly rather than require wholesale replacement. I wanted to provide SDKs and plugins that let teams connect ErrorZen with technologies they're already using, like Firebase Crashlytics or Sentry. Additionally, I needed a flexible API that businesses could use to customize integrations based on their specific workflows and requirements.

\subsubsection{User Interface and Access Management}

The dashboard needed to be intuitive enough that developers could immediately start viewing, filtering, and analyzing errors without extensive training. I wanted powerful search and filtering capabilities that let teams drill down to specific error patterns or timeframes. Security was equally important, so I needed robust role and permission management that ensured teams could only access data relevant to their projects while keeping sensitive information protected.

\subsection{Non-functional Requirements}

These needs concern system quality, performance and security.

\subsubsection{Performance and Scalability}

I set aggressive performance targets because slow error tracking tools defeat their own purpose. The system needed to handle massive volumes of logs -- thousands of errors per second during peak loads -- without degrading performance. I targeted sub-200ms response times for error retrieval because developers shouldn't wait when debugging production issues. The architecture also had to scale horizontally to support growing numbers of users and integrations without any performance degradation, which meant careful attention to database queries, caching strategies, and service communication patterns.

\subsubsection{Security and Compliance}

Given that error logs often contain sensitive information, security couldn't be an afterthought. I implemented AES-256 encryption for all sensitive user and error data, both in transit and at rest. Authentication and authorization had to be bulletproof, using industry-standard JWT tokens that could integrate with existing OAuth providers. Compliance with regulations like GDPR and standards like ISO 27001 was non-negotiable, especially for enterprise customers who need detailed audit trails and data governance capabilities.

\subsubsection{Availability and Reliability}

An error tracking system that goes down when you need it most is worse than useless, so I committed to maintaining 99.9\% uptime. I implemented comprehensive backup and recovery mechanisms so that data is never lost, even in catastrophic failure scenarios. Real-time monitoring of the platform itself was essential -- using health checks, metrics, and alerts to catch potential issues before they impact users. Essentially, the system needed to be more reliable than the applications it monitors.

\subsubsection{Compatibility and Integration}

Development teams work across diverse environments, so ErrorZen needed to function seamlessly on Linux, Windows, and macOS without platform-specific quirks. Language and framework compatibility was equally critical -- I wanted support for Python, Node.js, Java, Flutter, and other popular technologies so teams wouldn't need to change their stack to use the platform. I chose REST APIs as the primary communication protocol because they're universally understood and provide efficient, straightforward integration with any frontend technology.

\subsubsection{Ease of Use and Maintainability}

I believe tools should feel natural to use, not require extensive training manuals. The interface needed to be intuitive enough that developers could start using it productively within minutes of first login. Comprehensive documentation for APIs and SDKs was essential -- I've been frustrated too many times by undocumented or poorly documented tools. I also committed to providing ongoing technical support and regular updates, because a platform is only as good as the support behind it when users encounter issues or need new features.

\subsection{Constraints}

\begin{itemize}
\item \textbf{Time-to-Market:} The MVP must be delivered in 17 weeks to meet client onboarding deadlines. Rationale: Rapid Application Development (RAD) and Agile-Scrum methodologies will accelerate iterations.
\item \textbf{Offline-First Support:} Must cache and sync errors locally for mobile/remote developers with poor connectivity. Rationale: PostgreSQL's write-ahead logging (WAL) and Vue.js's local storage ensure data consistency.
\item \textbf{Open-Source Priority:} Prefer open-source tools (e.g., PostgreSQL, PyTorch) to minimise licensing costs.
\item \textbf{Multi-Platform SDKs:} SDKs must support Python, Node.js, Java, and mobile (iOS/Android) for broad compatibility.
\item \textbf{Zero Manual DevOps:} CI/CD pipelines (GitHub Actions/Jenkins) must fully automate testing/deployment without human intervention.
\end{itemize}

\subsection{System Diagrams}

\subsubsection{Use Case Diagram}

\begin{figure}[H]
\centering
\includegraphics[width=0.7\textwidth,keepaspectratio]{rapport/media/userscasglobal.png}
\caption{Use Case Diagram}
\label{fig:usecase}
\end{figure}

\subsubsection{Class Diagram of the System}

\begin{figure}[H]
\centering
\includegraphics[width=\textwidth,keepaspectratio]{rapport/media/diag_class_global.png}
\caption{Class Diagram of the System}
\label{fig:classdiagram}
\end{figure}

\subsubsection{Deployment Diagram}

\begin{figure}[H]
\centering
\includegraphics[width=0.6\textwidth,keepaspectratio]{rapport/media/diag_deploy.png}
\caption{Deployment Architecture}
\label{fig:components}
\end{figure}

\subsubsection{Requirement Traceability Matrix}

\begin{table}[H]
\centering
\caption{Requirement Traceability Matrix}
\footnotesize
\begin{tabular}{|p{1.5cm}|p{5cm}|p{2.5cm}|p{3cm}|p{2cm}|}
\hline
\textbf{Req. ID} & \textbf{Requirement Description} & \textbf{Source} & \textbf{Implementation Module} & \textbf{Status} \\
\hline
RQ-01 & The system must authenticate all users before access & Business Rule & Auth Module & Implemented \\
\hline
RQ-02 & Developer must handle errors & Functional Req. & Error Handling Service & In Testing \\
\hline
RQ-03 & Developer must handle analytics & Functional Req. & Analytics Service & Implemented \\
\hline
RQ-04 & AI agent must capture real-time errors & Functional Req. & AI Monitoring Module & Pending \\
\hline
RQ-05 & AI agent must suggest possible fixes & Functional Req. & AI Recommendation Engine & Planned \\
\hline
RQ-06 & System must generate bills automatically & Functional Req. & Billing Service & Implemented \\
\hline
RQ-07 & Admin must manage user roles & Functional Req. & User Management Module & Implemented \\
\hline
RQ-08 & Manager must monitor deployment processes & Functional Req. & Deployment Service & In Testing \\
\hline
\end{tabular}
\end{table}

\subsection{Summary}

This chapter defined the expected functionalities and performance characteristics of the system. These requirements guided the design and implementation of the application. The use of diagrams helped visualise user interactions and data flows clearly.