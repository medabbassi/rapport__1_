% Section 3: Literature Review / State of the Art

\section{Literature Review / State of the Art}

\subsection{Introduction}

Before implementing any technical solution, it is essential to review existing work, approaches, and technologies related to the problem being addressed. This review of the literature aims to provide an overview of similar systems, tools, and frameworks and to justify the technical choices made during this project.

\subsection{Existing Solutions}

Several platforms and tools have been developed to address error/bug logging and detection. Each offers different features and uses various technologies.

\subsubsection{Sentry}

\begin{figure}[H]
\centering
\includegraphics[width=0.8\textwidth,keepaspectratio]{rapport/media/image1.png}
\caption{Sentry Architecture Overview}
\label{fig:sentry_architecture}
\end{figure}

\begin{figure}[H]
\centering
\includegraphics[width=0.7\textwidth,keepaspectratio]{rapport/media/image3.png}
\caption{Sentry System Design}
\label{fig:sentry_design}
\end{figure}

Sentry excels at real-time error monitoring with wide language support (JavaScript, Python, Ruby, Java, Go, PHP, .NET), detailed error reports with stack traces, DevOps integrations (GitHub, Slack, Jira), and user-friendly UI with release tracking.

\textbf{Limitations:} Prohibitive costs for high-volume apps, limited free tier, maintenance-heavy self-hosting, lacks advanced APM, steep learning curve, narrow focus on errors only.

\subsubsection{Dynatrace}

\begin{figure}[H]
\centering
\includegraphics[width=0.8\textwidth,keepaspectratio]{rapport/media/image5.png}
\caption{Dynatrace Database Schema}
\label{fig:dynatrace_schema}
\end{figure}

Dynatrace is an enterprise-grade AI-powered observability platform with Davis AI for root cause analysis, full-stack monitoring (applications, microservices, containers, infrastructure), automatic dependency mapping, real user monitoring, and multi-cloud support (AWS, Azure, GCP).

\textbf{Limitations:} Extraordinarily expensive, complex setup, limited customization, resource-heavy, separate log management (Dynatrace Grail), and strong vendor lock-in.

\begin{figure}[H]
\centering
\includegraphics[width=0.75\textwidth,keepaspectratio]{rapport/media/image6.png}
\caption{Competitive Analysis: ErrorZen vs Existing Solutions}
\label{fig:comparison}
\end{figure}

\subsection{Comparison of Existing Solutions}

\begin{table}[H]
\centering
\caption{Comparison of Sentry vs Dynatrace}
\footnotesize
\begin{tabular}{|p{4.5cm}|p{4.5cm}|p{4.5cm}|}
\hline
\textbf{Feature/Capability} & \textbf{Sentry} & \textbf{Dynatrace} \\ \hline
Primary Use Case & Error \& Performance Monitoring & Full-Stack APM \& AI Observability \\ \hline
Scalability & Poor at large-scale event volumes & Highly scalable (enterprise-grade) \\ \hline
Offline Support & No offline error tracking & No offline monitoring \\ \hline
User Interface (UI) & Modern but simple & Powerful but complex \\ \hline
Key Missing Features & No infra/cloud monitoring & No built-in log management (Grail add-on) \\ \hline
Root Cause Analysis & Manual (basic traces) & AI-powered (Davis AI) \\ \hline
Real User Monitoring & Limited (frontend-focused) & Advanced (RUM + Synthetic) \\ \hline
Performance Monitoring & Basic (transactions, latency) & Full APM (code-level, DB, infra) \\ \hline
Cloud/Serverless & Limited & AWS Lambda, Azure Functions, etc. \\ \hline
Cost & Affordable for startups & Very expensive (enterprise pricing) \\ \hline
Best For & Dev teams need error tracking & Enterprises needing AI-driven APM \\ \hline
\end{tabular}
\end{table}

\subsection{Why ErrorZen Will Outperform Existing Solutions}

After thoroughly analyzing Sentry and Dynatrace, I identified critical gaps that existing tools don't adequately address. While these platforms excel in specific areas -- Sentry for straightforward error tracking, Dynatrace for comprehensive APM -- they share fundamental limitations that frustrated me as a developer. Automation is limited to detection and alerting; you still manually triage issues and write fixes. Scalability becomes problematic when dealing with high-frequency errors, especially given Sentry's pricing model. DevOps and CI/CD integration requires manual intervention at key points rather than being truly seamless. Even Dynatrace's vaunted AI only analyzes and alerts -- it doesn't actually remediate issues, which delays resolution.

I designed ErrorZen specifically to solve these challenges in ways existing tools don't. The AI-powered auto-correction capability is fundamentally different from what Sentry or Dynatrace offer. While Sentry requires manual debugging and Dynatrace only provides AI alerts, ErrorZen uses machine learning to proactively generate and apply fixes, dramatically reducing mean time to resolution. I built end-to-end DevOps automation that integrates directly with CI/CD pipelines to automatically test and deploy patches without manual intervention -- this eliminates steps that neither Dynatrace nor Sentry can address. ErrorZen provides unified cross-platform monitoring that tracks frontend, backend, and mobile in one coherent dashboard, while competitors tend to silo data. Sentry lacks infrastructure insights, and Dynatrace's comprehensive view comes at enterprise prices. Real-time notifications combine Slack and email alerts with actionable fixes, going beyond Dynatrace's passive alerts or Sentry's basic notifications. Finally, ErrorZen's architecture scales cost-effectively, avoiding both Dynatrace's prohibitive enterprise pricing and Sentry's volume-based limits through optimized event processing.

\subsection{Technology Selection}

\textbf{Backend:} Go (concurrency, performance), Python (AI integration)\\
\textbf{Data:} PostgreSQL (ACID compliance), MongoDB (flexible logs)\\
\textbf{APIs:} gRPC (internal, low-latency), REST (external, simplicity)\\
\textbf{Frontend:} Vue.js (reactive, real-time dashboards)\\
\textbf{AI:} DeepSeek API (sophisticated analysis without ML overhead)\\
\textbf{DevOps:} GitHub Actions (primary), Jenkins (legacy support)\\
\textbf{Infrastructure:} Docker + Kubernetes (auto-scaling), AWS + GCP (global reliability)

\begin{figure}[H]
\centering
\includegraphics[width=0.7\textwidth,keepaspectratio]{rapport/media/grpc-icon-color.png}
\caption{gRPC Communication Architecture}
\label{fig:grpc}
\end{figure}

\subsection{Summary}

Conducting this literature and technology review was invaluable for understanding where ErrorZen needed to fit in the ecosystem. I gained clear insights into the current market state, recognizing both the strengths of established players and the gaps they leave unfilled. Understanding common limitations in existing systems helped me avoid repeating their mistakes while building on their successes. Researching best practices in selecting modern, scalable technologies informed every architectural decision I made throughout the project. This foundational research ensured I built a solution that's not only technically sound but also aligned with real-world needs that developers actually experience in production environments.